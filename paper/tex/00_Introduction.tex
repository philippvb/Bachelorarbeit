\chapter{Introduction}
In recent years, deep neural networks have created huge success in a variety of
tasks ranging from image classification to autonomous driving. However when
viewed from a mathematical standpoint, this success is quite surprising. In
contrast to traditional optimization, optimization in deep neural networks faces
some novel challenges.

One major problem is the loss landscape of neural networks. In traditional
machine learning, the loss landscape is usually convex and only has one minima,
which is therefore the global one. However in deep neural networks, there exist
a large number of local minima. Therefore, finding a minimum does not guarantee
to achieve optimal performance. In addition, the local structure of the loss
landscape is not resembling the global structure, meaning that the path taken by
optimizers mostly is suboptimal and leads away from the global optimum. The high
number of parameters and the loss landscape which is expensive to compute
further imped the task.

Despite these challenges, even simple optimizers like Stochastic Gradient
Descent lead to string performances. Recent literature suggests this might
happen because most local minima indeed have good performance
\cite{choromanska2015loss}. In addition, searching for a global minima might not
be useful, as this quickly leads to overfitting, where the networks memorizes
the training set rather than learning the data distribution. Currently, it seems
like the local minima are arranged in cells, where each minima of one cell has
the same loss and all are connected by areas of low loss \cite{he2020piecewise},
\cite{draxler2018essentially}. Furthermore, the cells are surrounnded by areas
of high loss. Therefore when an optimizer reaches one of these cells, it is
unlikely to move out of it again. This means, that during training, only a small
part of the loss landscape is explored.

Even if minima in other areas might have approximatly the same cost, it can be
beneficial to explore the loss landscape further. The work of Loshchilov \&
Hutter \cite{loshchilov2016sgdr} realized this by using warm restarts. Here, the
learning rate is set back to the initial learning rate after a fixed number of
epochs. Therefore, the update steps got larger and the landscape was explored
further. This lead to an increase in maximum performance after the learning rate
was decayed again.

Inspired by their results, the goal of this paper is to pursue the idea further.
Rather than letting the exploration happen implicity by a high learning rate,
this work tries to force exploration more directly by rewarding moving away from
fixed locations on the loss landscape. The hope beeing, that new areas might
lead to an increase in performance, even if it could be small. As different
locations on the loss landscape should mean different behaviours of the network,
further fields of research like Ensemble methods could possibly benefit from the
approach.

