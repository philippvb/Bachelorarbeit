Current research suggests that most local minima on the loss landscape of deep
neural networks are of low loss and interconnected by paths
\cite{choromanska2015loss} \cite{draxler2018essentially} \cite{he2020piecewise}.
Nevertheless, Cosine Decay with Warm Restart \cite{loshchilov2016sgdr} has shown
that it can be beneficial to explore the loss landscape further. This work
follows up on the idea, but rather forces the exploration more directly by the
addition of a distance term to the loss function. Visually, the distance
function places a hill at the current position, called checkpoint, on the loss
landscape. This approach leads to further distance from the checkpoint than
without the distance function. In general, the effect does not translate to
validation accuracy, which seems to confirm the view, that areas of low loss
exist everywhere and are of same cost. For the addition of multiple checkpoints,
there was a small increasement similar to Cosine Decay with Warm Restarts,
probably to the same effect on the gradient. The distance term also stabilized
training with schedulers for a suboptimal initial learning rate. Finally, the
idea was tested in combination with snapshot ensembling, to reduce the
correlation between the snapshots. However further distance did not imply lower
correlation, which questions the connection between the loss landscape and
behaviour of the network.