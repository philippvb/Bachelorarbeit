\chapter{Results}
Even with such a simple formulation, there are many different variations that
can be tested. First is the influence of the strength and width parameter of the
distance function, which will be tested in section \ref{res:Hyperparameters}. We
will also briefly test different kernels besides the RBF Kernel from equation
\ref{eq:RBF} in section \ref{res:Kernel}. In section \ref{res:Multiple}, we will
use more than one checkpoint and will also try to combine these to a global one.
The interaction of training hyperparameters like learning rate and epochs will
be discussed in section \ref{res:Training}. Finally, the usage for ensemble
methods will be tested.

\section{Baseline}
\subsection{Base Network} % get better name
For the baseline, we use the hyperparameters as defined in section
\ref{sub:Hyperparameters}. Figure \ref{fig:MobileNetV2_baseline} shows the
result, if we add a checkpoint after 100 epochs of training. For the validation
accuray, we can see that it stays similar to the accuracy without the distance
function. If we plot the distance the network gets to the checkpoint however, we
can see that the network trained with distance function clearly increases its
distance more than the one trained without. Furthermore, the red plot follows
the shape we would expect from the distance function. At the beginning, the
distance increases relatively fast, as can be expected from the high value of
the RBF Kernel. But as the distance increases, its gradient becomes smaller,
similar to the gradient of the Kernel. The blue plot in contrast follows an
almost linear curve, whith a much smaller gradient. Therefore, the additional
term succeeds in the initial goal, which was to distance from the checkpoint.

\begin{figure}[h]\label{fig:MobileNetV2_baseline}
    \begin{center}
        \begin{tikzpicture}
            \begin{groupplot}[
                group style={
                group size=2 by 1,
                horizontal sep=10pt,
                group name=G},
                width=8cm,
            ]

            \nextgroupplot[
            grid=major, 
            grid style={dashed,gray!30},
            xlabel=Epoch, % Set the labels
            ylabel=Validation Accuracy,
            ymin=0.8]
            \addplot[mark=None, color=red] 
                table[x=Step, y=Value, col sep=comma]{images/network_csv/baseline/baseline/MobileNetV2_baseline_validation_acuracy.csv};
            \addplot[mark=None, color=blue] 
                table[x=Step, y=Value, col sep=comma]{images/network_csv/baseline/baseline_distance/MobileNetV2_baseline_distance_validation_acuracy.csv};
            
            \nextgroupplot[
                grid=major, 
                grid style={dashed,gray!30},
                xlabel=Epoch, % Set the labels
                ylabel=Distance,
                yticklabel pos=right,
                ylabel near ticks]
                \addplot[mark=None, color=red] 
                    table[x=Step, y=Value, col sep=comma]{images/network_csv/baseline/baseline/MobileNetV2_baseline_distance0.csv};
                \addplot[mark=None, color=blue] 
                    table[x=Step, y=Value, col sep=comma]{images/network_csv/baseline/baseline_distance/MobileNetV2_baseline_distance_distance0.csv};
    
            \end{groupplot}
        \end{tikzpicture}
        \caption{Validation accuray and Distance values for a network trained without distance function (red) and with distance function (blue).}
    \end{center}
\end{figure}


scheduler
\begin{figure}[h]\label{fig:MobileNetV2_baseline_scheduler}
    \begin{center}
        \begin{tikzpicture}
            \begin{groupplot}[
                group style={
                group size=2 by 1,
                horizontal sep=10pt,
                group name=G},
                width=8cm,
            ]

            \nextgroupplot[
            grid=major, 
            grid style={dashed,gray!30},
            xlabel=Epoch, % Set the labels
            ylabel=Validation Accuracy,
            ymin=0.8]
            \addplot[mark=None, color=red] 
                table[x=Step, y=Value, col sep=comma]{images/network_csv/baseline/baseline_scheduler/MobileNetV2_baseline_scheduler_validation_acuracy.csv};
            \addplot[mark=None, color=blue] 
                table[x=Step, y=Value, col sep=comma]{images/network_csv/baseline/baseline_scheduler_distance/MobileNetV2_baseline_scheduler_distance_validation_acuracy.csv};
            
            \nextgroupplot[
                grid=major, 
                grid style={dashed,gray!30},
                xlabel=Epoch, % Set the labels
                ylabel=Distance,
                yticklabel pos=right,
                ylabel near ticks]
                \addplot[mark=None, color=red] 
                    table[x=Step, y=Value, col sep=comma]{images/network_csv/baseline/baseline_scheduler/MobileNetV2_baseline_scheduler_distance0.csv};
                \addplot[mark=None, color=blue] 
                    table[x=Step, y=Value, col sep=comma]{images/network_csv/baseline/baseline_scheduler_distance/MobileNetV2_baseline_scheduler_distance_distance0.csv};
    
            \end{groupplot}
        \end{tikzpicture}
        \caption{Validation accuray and Distance values for a network with a step scheduler and warm restarts. The learning rate is dropped after 50 epochs and restarted after 150.
        The red curve shows a network trained without distance function and the blue curve with distance function.}
    \end{center}
\end{figure}






\section{Distance function Hyperparameters}\label{res:Hyperparameters}
Recall the formulation of the distance function from equation
\ref{eq:DistanceFinal}:
\begin{align}
    k(\theta_1, \theta_2)=exp(-\frac{d(\theta_1, \theta_2)}{2\sigma^2})
\end{align}
and the total combination with the ordinary loss function from equation
\ref{eq:LossDistance}:
\begin{align}
    L=\sum_{c} \delta_{yc} log(f(x)_c) + w \cdot distance(\theta, \theta_c)
\end{align}
Here, we can see the two hyperparameters width $\sigma$ and weight $w$, whose
influence will be investigated in this chapter.
\subsection{weight}

\subsection{width}
We have seen in chapter \ref{distance_function}, how the width $\sigma$
influences the distance function: The larger $\sigma$ gets, the wider it
becomes. Therefore, a larger $\sigma$ should result in the network distancing
more from it's checkpoint than for smaller values. 






\section{Kernel}\label{res:Kernel}

\section{Regularization}

\section{Multiple Minimas}\label{res:Multiple}
\subsection{multiple}
\subsection{merge}

\section{Training Hyperparameters}\label{res:Training}
\subsection{lr}\label{res:Learning rate}
wrong lr and how scheduler is used
\subsection{epochs}\label{res:Epochs}
epochs and epoch time

\section{ensemble methods}



example picture:
\begin{figure}[h]\label{fig:MobileNetV2_baseline}
    \begin{center}
        \begin{tikzpicture}
            \begin{groupplot}[
                group style={
                group size=2 by 1,
                horizontal sep=10pt,
                group name=G},
                width=8cm,
            ]

            \nextgroupplot[
            grid=major, 
            grid style={dashed,gray!30},
            xlabel=Epoch, % Set the labels
            ylabel=Validation Accuracy,
            ymin=0.8]
            \addplot[mark=None, color=red] 
                table[x=Step, y=Value, col sep=comma]{images/network_csv/baseline/baseline/MobileNetV2_baseline_validation_acuracy.csv};
            \addplot[mark=None, color=blue] 
                table[x=Step, y=Value, col sep=comma]{images/network_csv/baseline/baseline_distance/MobileNetV2_baseline_distance_validation_acuracy.csv};
            
            \nextgroupplot[
                grid=major, 
                grid style={dashed,gray!30},
                xlabel=Epoch, % Set the labels
                ylabel=Distance,
                yticklabel pos=right,
                ylabel near ticks]
                \addplot[mark=None, color=red] 
                    table[x=Step, y=Value, col sep=comma]{images/network_csv/baseline/baseline/MobileNetV2_baseline_distance0.csv};
                \addplot[mark=None, color=blue] 
                    table[x=Step, y=Value, col sep=comma]{images/network_csv/baseline/baseline_distance/MobileNetV2_baseline_distance_distance0.csv};
    
            \end{groupplot}
        \end{tikzpicture}
        \caption{Validation accuray (left) and Distance values (right) for a network trained without distance function (red) and with distance function (blue).}
    \end{center}
\end{figure}