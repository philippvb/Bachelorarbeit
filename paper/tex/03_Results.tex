\chapter{Results}
Even with such a simple formulation, there are many different variations that
can be tested. First is the influence of the strength and width parameter of the
distance function, which will be tested in section \ref{res:Hyperparameters}. We
will also briefly test different kernels besides the RBF Kernel from equation
\ref{eq:RBF} in section \ref{res:Kernel}. In section \ref{res:Multiple}, we will
use more than one checkpoint and will also try to combine these to a global one.
The interaction of training hyperparameters like learning rate and epochs will
be discussed in section \ref{res:Training}. Finally, the usage for ensemble
methods will be tested.

\section{Baseline}
For the baseline, we use the hyperparameters as defined in section
\ref{sub:Hyperparameters}.






\section{Distance function Hyperparameters}\label{res:Hyperparameters}
Recall the formulation of the distance function from equation
\ref{eq:DistanceFinal}:
\begin{align}
    k(\theta_1, \theta_2)=exp(-\frac{d(\theta_1, \theta_2)}{2\sigma^2})
\end{align}
and the total combination with the ordinary loss function from equation
\ref{eq:LossDistance}:
\begin{align}
    L=\sum_{c} \delta_{yc} log(f(x)_c) + w \cdot distance(\theta, \theta_c)
\end{align}
Here, we can see the two hyperparameters width $\sigma$ and weight $w$, whose
influence will be investigated in this chapter.
\subsection{weight}

\subsection{width}
We have seen in chapter \ref{distance_function}, how the width $\sigma$
influences the distance function: The larger $\sigma$ gets, the wider it
becomes. Therefore, a larger $\sigma$ should result in the network distancing
more from it's checkpoint than for smaller values.






\section{Kernel}\label{res:Kernel}

\section{Regularization}

\section{Multiple Minimas}\label{res:Multiple}
\subsection{multiple}
\subsection{merge}

\section{Training Hyperparameters}\label{res:Training}
\subsection{lr}\label{res:Learning rate}
wrong lr and how scheduler is used
\subsection{epochs}\label{res:Epochs}
epochs and epoch time

\section{ensemble methods}



example picture:
\begin{figure}[h]\label{fig:test}
    \begin{center}
        \begin{tikzpicture}
            \begin{axis}
                [width=\linewidth, % Scale the plot to \linewidth
                grid=major, 
                grid style={dashed,gray!30},
                xlabel=X Axis, % Set the labels
                ylabel=Y Axis,
                ymin=0.8]
                \addplot[mark=None, color=red] 
                table[x=Step, y=Value, col sep=comma]{images/run-mobileNetV2_ensemble_distance-tag-Ensemble_accuracy.csv};
            \end{axis}
        \end{tikzpicture}
        \caption{My first autogenerated plot.}
    \end{center}
\end{figure}