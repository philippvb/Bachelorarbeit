\chapter{Results}
Even with such a simple formulation, there are many different variations that
can be tested. First is the influence of the strength and width parameter of the
distance function, which will be tested in section \ref{res:Hyperparameters}. We
will also briefly test different kernels besides the RBF Kernel from equation
\ref{eq:RBF} in section \ref{res:Kernel}. In section \ref{res:Multiple}, we will
use more than one checkpoint and will also try to combine these to a global one.
The interaction of training hyperparameters like learning rate and epochs will
be discussed in section \ref{res:Training}. Finally, the usage for ensemble
methods will be tested.

\section{Baseline}
For the baseline, we use the hyperparameters as defined in section
\ref{sub:Hyperparameters}. Figure \ref{fig:Results_baseline} shows the result,
if we add a checkpoint after 100 epochs of training. For the validation accuray,
we can see that it stays similar to the accuracy without the distance function.
If we plot the distance the network gets to the checkpoint however, we can see
that the network trained with distance function clearly increases its distance
more than the one trained without. Furthermore, the red plot follows the shape
we would expect from the distance function. At the beginning, the distance
increases relatively fast, as can be expected from the high gradient of the RBF
Kernel. But as the distance increases, its gradient becomes smaller, similar to
the gradient of the Kernel. The blue plot in contrast follows an almost linear
curve, whith a much smaller gradient. Therefore, the additional term succeeds in
the initial goal, which was to distance from the checkpoint. Note that despite
the validation accuracy beeing in an area of convergence, SGD even without
distance function keeps on walking and never truly stops at a point.

\begin{figure}[h]\label{fig:Results_baseline}
    \begin{center}
        \begin{tikzpicture}
            \begin{groupplot}[
                group style={
                group size=2 by 2,
                horizontal sep=10pt,
                vertical sep=10pt,
                group name=G},
                width=8cm,
            ]

            \nextgroupplot[
            grid=major, 
            grid style={dashed,gray!30},
            % xlabel=Epoch,
            ylabel=Validation Accuracy,
            xticklabels={,,},
            ymin=0.8,
            xmin=-10]
            \addplot[mark=None, color=red] 
                table[x=Step, y=Value, col sep=comma]{images/network_csv/baseline/baseline/MobileNetV2_baseline_validation_acuracy.csv};
            \addplot[mark=None, color=blue] 
                table[x=Step, y=Value, col sep=comma]{images/network_csv/baseline/baseline_distance/MobileNetV2_baseline_distance_validation_acuracy.csv};
            
            \nextgroupplot[
                grid=major, 
                grid style={dashed,gray!30},
                % xlabel=Epoch,
                ylabel=Distance,
                yticklabel pos=right,
                xticklabels={,,},
                ylabel near ticks]
                \addplot[mark=None, color=red] 
                    table[x=Step, y=Value, col sep=comma]{images/network_csv/baseline/baseline/MobileNetV2_baseline_distance0.csv};
                \addplot[mark=None, color=blue] 
                    table[x=Step, y=Value, col sep=comma]{images/network_csv/baseline/baseline_distance/MobileNetV2_baseline_distance_distance0.csv};
    

            \nextgroupplot[
            grid=major, 
            grid style={dashed,gray!30},
            xlabel=Epoch, % Set the labels
            ylabel=Validation Accuracy,
            ymin=0.8,
            xmin=-10]
            \addplot[mark=None, color=red] 
                table[x=Step, y=Value, col sep=comma]{images/network_csv/baseline/baseline/ResNet32_baseline_validation_acuracy.csv};
            \addplot[mark=None, color=blue] 
                table[x=Step, y=Value, col sep=comma]{images/network_csv/baseline/baseline_distance/ResNet32_baseline_distance_validation_acuracy.csv};
            
            \nextgroupplot[
                grid=major, 
                grid style={dashed,gray!30},
                xlabel=Epoch, % Set the labels
                ylabel=Distance,
                yticklabel pos=right,
                ylabel near ticks]
                \addplot[mark=None, color=red] 
                    table[x=Step, y=Value, col sep=comma]{images/network_csv/baseline/baseline/ResNet32_baseline_distance0.csv};
                \addplot[mark=None, color=blue] 
                    table[x=Step, y=Value, col sep=comma]{images/network_csv/baseline/baseline_distance/ResNet32_baseline_distance_distance0.csv};

            \end{groupplot}
        \end{tikzpicture}
        \caption{Validation accuray and Distance for MobileNetV2 (upper) and ResNet32 (lower) trained without distance function (red) and with distance function (blue).}
    \end{center}
\end{figure}







\section{Distance function Hyperparameters}\label{res:Hyperparameters}
\subsection{strength}
Recall section \ref{eq:LossDistance}, where we added strength as a
hyperparameter, controlling the influence of the distance function. As the RBF
Kernel is bound between 0 and 1, the strength hyperparameter will increase that
bound between 0 and the strength value.

For higher strength values, the validation accuracy receives an initial drop,
which is larger the higher the strength parameter value. This may be due to the
increased influence of the strength function. Each distance between the
parameters and the checkpoint starts at 0, therefore the values of the
exponential function starts at $strength \cdot 1$. A higher strength value will
lead to a higher influence on the loss function, and consequently the gradient.
The gradient of the cross-entropy loss will be insignificant, and the optimizer
will update the weights without regards to the validation accuracy, hence the
drop at the beginning. The distance plot reflects this behaviour, as we can see
an larger increase in distance for larger strength values.

After the initial step however, the distance quickly converges. This is due to
the fact that after the initial increase, the value of the Kernel quickly
reaches 0. [TODO: Sow plot of strength]. Therefore, the distance Kernel becomes
insignificant again, and the Cross-Entropy loss is followed. The validation
accuracy reflects this behaviour, as after the initial drop the accuracy
recovers. Interestingly, the validation accuracy doesn't reach the level without
distance function. An possible explanation would be, that the network has left
the area of low loss, and is not able to recover again.[Mayvbe to
undiffernetuial boundaries or beeing stuck in that cell.] The question that now
arises is if there is no other area of high accuracy other than the inital
convergence. If one exists, one of the runs should converge here. If none exist,
the question is why SGD finds this area. [Mabe due to the weights having to be
small???, low dependencies and high generalization??] 


basic results:
- if higher strength factor, distance rises more quickly
- then stays in this area of distance
- validation accuracy drops because first influenence of distance
- however never recovers - areas of bad accuracy??




\begin{figure}[h]\label{fig:Results_strength}
    \begin{center}
        \begin{tikzpicture}
            \begin{groupplot}[
                group style={
                group size=2 by 2,
                horizontal sep=10pt,
                vertical sep=10pt,
                group name=G},
                width=8cm
            ]

            \nextgroupplot[
            grid=major, 
            grid style={dashed,gray!30},
            % xlabel=Epoch,
            ylabel=Validation Accuracy,
            xticklabels={,,},
            ymin=0.6,
            xmin=-10]
            \addplot[mark=None, color=red] 
                table[x=Step, y=Value, col sep=comma]{images/network_csv/strength/MobileNetV2/MobileNetV2_strength_e1_validation_acuracy.csv};
            \addplot[mark=None, color=blue] 
                table[x=Step, y=Value, col sep=comma]{images/network_csv/baseline/baseline_distance/MobileNetV2_baseline_distance_validation_acuracy.csv};
            \addplot[mark=None, color=orange] 
                table[x=Step, y=Value, col sep=comma]{images/network_csv/strength/MobileNetV2/MobileNetV2_strength_e2_validation_acuracy.csv};
            \addplot[mark=None, color=green] 
                table[x=Step, y=Value, col sep=comma]{images/network_csv/strength/MobileNetV2/MobileNetV2_strength_e3_validation_acuracy.csv};
            
            \nextgroupplot[
                grid=major, 
                grid style={dashed,gray!30},
                % xlabel=Epoch,
                ylabel=Distance,
                yticklabel pos=right,
                xticklabels={,,},
                ylabel near ticks]
            \addplot[mark=None, color=red] 
                table[x=Step, y=Value, col sep=comma]{images/network_csv/strength/MobileNetV2/MobileNetV2_strength_e1_distance0.csv};
            \addplot[mark=None, color=blue] 
                table[x=Step, y=Value, col sep=comma]{images/network_csv/baseline/baseline_distance/MobileNetV2_baseline_distance_distance0.csv};
            \addplot[mark=None, color=orange] 
                table[x=Step, y=Value, col sep=comma]{images/network_csv/strength/MobileNetV2/MobileNetV2_strength_e2_distance0.csv};
            \addplot[mark=None, color=green] 
                table[x=Step, y=Value, col sep=comma]{images/network_csv/strength/MobileNetV2/MobileNetV2_strength_e3_distance0.csv};
    

            \nextgroupplot[
            grid=major, 
            grid style={dashed,gray!30},
            xlabel=Epoch, % Set the labels
            ylabel=Validation Accuracy,
            ymin=0.6,
            xmin=-10]
            \addplot[mark=None, color=red] 
                table[x=Step, y=Value, col sep=comma]{images/network_csv/strength/ResNet32/ResNet32_strength_e1_validation_acuracy.csv};
            \addplot[mark=None, color=blue] 
                table[x=Step, y=Value, col sep=comma]{images/network_csv/baseline/baseline_distance/ResNet32_baseline_distance_validation_acuracy.csv};
            \addplot[mark=None, color=orange] 
                table[x=Step, y=Value, col sep=comma]{images/network_csv/strength/ResNet32/ResNet32_strength_e2_validation_acuracy.csv};
            \addplot[mark=None, color=green] 
                table[x=Step, y=Value, col sep=comma]{images/network_csv/strength/ResNet32/ResNet32_strength_e3_validation_acuracy.csv};
            
            \nextgroupplot[
                grid=major, 
                grid style={dashed,gray!30},
                xlabel=Epoch, % Set the labels
                ylabel=Distance,
                yticklabel pos=right,
                ylabel near ticks]
            \addplot[mark=None, color=red] 
                table[x=Step, y=Value, col sep=comma]{images/network_csv/strength/ResNet32/ResNet32_strength_e1_distance0.csv};
            \addplot[mark=None, color=blue] 
                table[x=Step, y=Value, col sep=comma]{images/network_csv/baseline/baseline_distance/ResNet32_baseline_distance_distance0.csv};
            \addplot[mark=None, color=orange] 
                table[x=Step, y=Value, col sep=comma]{images/network_csv/strength/ResNet32/ResNet32_strength_e2_distance0.csv};
            \addplot[mark=None, color=green] 
                table[x=Step, y=Value, col sep=comma]{images/network_csv/strength/ResNet32/ResNet32_strength_e3_distance0.csv};

            \end{groupplot}
        \end{tikzpicture}
        \caption{Validation accuray and Distance for MobileNetV2 (upper) and ResNet32 (lower) trained without distance function (red) and with distance function (blue).}
    \end{center}
\end{figure}

\subsection{width}
We have seen in chapter \ref{distance_function}, how the width $\sigma$
influences the distance function: The larger $\sigma$ gets, the wider it
becomes. Therefore, a larger $\sigma$ should result in the network distancing
further from it's checkpoint than for smaller values. That's exactly what can be
seen in figure \ref{}: For an $\sigma$ of 0.1, the distance follows the one of
network trained without distance function, due to the RBF Kernel quickly
becoming 0. The larger the width, the slower the values of the RBF Kernel
decrease for further distance. Therefore, the gradient of the distance function
will stay quite large, pushing the network further away. This can be seen for
higher $\sigma$ values, where the distance increases further.


Initially however, the distance plot doesn't reflect this behaviour. The curve
for $\sigma = 0.01$ has a higher distance value than for $\sigma = 0.001$.
That's because the gradient of the RBF Kernel is higher for small distances, the
smaller $\sigma$. But for larger distances, this effect flips with the gradient
of the larger width staying higher for more epochs. Therefore, the curves
intersect after additional epochs. The smaller width converges, while the larger
stays constant until eventually reaching it`s area of convergence. The
intersection happens later the larger the width, for $\sigma = 0.0001$, only
after 207 epochs.

This behaviour however has no influence on the validation accuracy, which stays
the same for every width, in contrast to the strength. This may be due to the
fact, that a larger width does in fact decreases the size of the gradient. A
smaller gradient over more epochs will be added to the gradient of the
Cross-Entropy Loss. The optimizer therefore nearly keeps on following the normal
gradient and descending the valley of low loss, with a small but steady push to
increase the distance to the checkpoint. Therefore, even if the network is
allowed to move quite freely on the loss surface, it finds areas of low loss and
high accuracy in distant areas of the checkpiont, meaning there has to be good
local minima everywhere on the landscape.









results: - wider kernel leads to higher value of
distance - maybe slower due to smaller gradient - no difference in validation
accuracy - if so, should prove that in every area exists low loss regions -
accuracy doesn't decrease -> there are valleys leading away from current point
with high accuracy - 







\section{Multiple Minimas}\label{res:Multiple}
idea: distance from multiple rather than 1 place
\subsection{multiple}
\subsection{merge}

\section{Training Hyperparameters}\label{res:Training}
\subsection{lr}\label{res:Learning rate}    
wrong lr and how scheduler is used

result idea:
- difference with wrong lr
- too high doenst lead to convergence, too low leads to slower one
- if kernel is applied, starts distancing
- 


for scheduler:
- maybe sgd walks in area of low accuracy, cannot recover (what is this area)
- distance kernel helps escape from this area


\subsection{epochs}\label{res:Epochs}
epochs and epoch time

maybe not use since or only for scheduler and multiple, since with one not helpful

\section{ensemble methods}
- idea: uncorellated networks
- also make run without warm restart

results:
- first show higher accuracy than single network
- than see, that distance has no better value than without
- show that distancing doenst mean more diverse prediction
- implication for loss landscape -> everywhere the network behaves the same??



example picture:
\begin{figure}[h]\label{fig:MobileNetV2_baseline}
    \begin{center}
        \begin{tikzpicture}
            \begin{groupplot}[
                group style={
                group size=2 by 1,
                horizontal sep=10pt,
                group name=G},
                width=8cm,
            ]

            \nextgroupplot[
            grid=major, 
            grid style={dashed,gray!30},
            xlabel=Epoch, % Set the labels
            ylabel=Validation Accuracy,
            ymin=0.8]
            \addplot[mark=None, color=red] 
                table[x=Step, y=Value, col sep=comma]{images/network_csv/baseline/baseline/MobileNetV2_baseline_validation_acuracy.csv};
            \addplot[mark=None, color=blue] 
                table[x=Step, y=Value, col sep=comma]{images/network_csv/baseline/baseline_distance/MobileNetV2_baseline_distance_validation_acuracy.csv};
            
            \nextgroupplot[
                grid=major, 
                grid style={dashed,gray!30},
                xlabel=Epoch, % Set the labels
                ylabel=Distance,
                yticklabel pos=right,
                ylabel near ticks]
                \addplot[mark=None, color=red] 
                    table[x=Step, y=Value, col sep=comma]{images/network_csv/baseline/baseline/MobileNetV2_baseline_distance0.csv};
                \addplot[mark=None, color=blue] 
                    table[x=Step, y=Value, col sep=comma]{images/network_csv/baseline/baseline_distance/MobileNetV2_baseline_distance_distance0.csv};
    
            \end{groupplot}
        \end{tikzpicture}
        \caption{Validation accuray (left) and Distance values (right) for a network trained without distance function (red) and with distance function (blue).}
    \end{center}
\end{figure}




have to do longer run of width and longer run of strength for highest factor at least, let distance kernel out again and look if comes back, add distance kernel for 0 epoch to show weight size