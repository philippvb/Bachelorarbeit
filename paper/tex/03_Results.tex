\chapter{Results}
After the theoretical introduction in chapter \ref{cha:Methods}, this chapter
provides some empirical results. In section \ref{res:baseline}, a baseline for
further comparisons will be created. Section \ref{res:Hyperparameters} will
focus on the effects of the hyperparameters of the distance function. How
multiple checkpoints can be used will section \ref{res:Multiple}show, followed
by the combination with learning rate schedulers \ref{res:Learning_rate}. We
will investigate the effects on ensemble methods in section \ref{res:Ensemble}
and finally study the computational cost in \ref{Res:Computational_cost}.



\section{Baseline}\label{res:Baseline} For the baseline, we use the
hyperparameters as defined in section \ref{sub:Hyperparameters}. First we train
both networks without distance function. After 100 epochs, we add a checkpoint
in the case of the network with distance function.

For the validation accuray, we can see no impact of the distance function.Both
networks show a standard learning curve, with a initial strong increase in
validation accuracy until coming into convergence in later epochs. This results
in a validation accuracy of around 90\% for MobileNetV2 and 89\% for ResNet32
after 600 epochs.




If we plot the distance the network gets to the checkpoint however, we can see
that the network trained with distance function clearly increases its distance
more than the one trained without. Furthermore, the red plot follows the shape
we would expect from the distance function. At the beginning, the distance
increases relatively fast, as can be expected from the high gradient of the RBF
Kernel. But as the distance increases, its gradient becomes smaller, similar to
the gradient of the Kernel. The blue plot in contrast follows an almost linear
curve, whith a much smaller gradient. Therefore, the additional term succeeds in
the initial goal, which was to distance from the checkpoint. Note that despite
the validation accuracy beeing in an area of convergence, SGD even without
distance function keeps on walking and never truly stops at a point.

\begin{figure}[h]\label{fig:Results_baseline}
    \begin{center}
        \begin{tikzpicture}
            \begin{groupplot}[
                group style={
                group size=2 by 2,
                horizontal sep=10pt,
                vertical sep=10pt,
                group name=G},
                width=8cm,
            ]

            \nextgroupplot[
            grid=major, 
            grid style={dashed,gray!30},
            % xlabel=Epoch,
            ylabel=Validation Accuracy,
            xticklabels={,,},
            ymin=0.8,
            xmin=-10]
            \addplot[mark=None, color=red] 
                table[x=Step, y=Value, col sep=comma]{images/network_csv/baseline/MobileNetV2/MobileNetV2_baseline_validation_acuracy.csv};
            \addplot[mark=None, color=blue] 
                table[x=Step, y=Value, col sep=comma]{images/network_csv/baseline/MobileNetV2/MobileNetV2_baseline_distance_validation_acuracy.csv};
            
            \nextgroupplot[
                grid=major, 
                grid style={dashed,gray!30},
                % xlabel=Epoch,
                ylabel=Distance,
                yticklabel pos=right,
                xticklabels={,,},
                ylabel near ticks]
                \addplot[mark=None, color=red] 
                    table[x=Step, y=Value, col sep=comma]{images/network_csv/baseline/MobileNetV2/MobileNetV2_baseline_distance0.csv};
                \addplot[mark=None, color=blue] 
                    table[x=Step, y=Value, col sep=comma]{images/network_csv/baseline/MobileNetV2/MobileNetV2_baseline_distance_distance0.csv};
    

            \nextgroupplot[
            grid=major, 
            grid style={dashed,gray!30},
            xlabel=Epoch, % Set the labels
            ylabel=Validation Accuracy,
            ymin=0.8,
            xmin=-10]
            \addplot[mark=None, color=red] 
                table[x=Step, y=Value, col sep=comma]{images/network_csv/baseline/ResNet32/ResNet32_baseline_validation_acuracy.csv};
            \addplot[mark=None, color=blue] 
                table[x=Step, y=Value, col sep=comma]{images/network_csv/baseline/ResNet32/ResNet32_baseline_distance_validation_acuracy.csv};
            
            \nextgroupplot[
                grid=major, 
                grid style={dashed,gray!30},
                xlabel=Epoch, % Set the labels
                ylabel=Distance,
                yticklabel pos=right,
                ylabel near ticks]
                \addplot[mark=None, color=red] 
                    table[x=Step, y=Value, col sep=comma]{images/network_csv/baseline/ResNet32/ResNet32_baseline_distance0.csv};
                \addplot[mark=None, color=blue] 
                    table[x=Step, y=Value, col sep=comma]{images/network_csv/baseline/ResNet32/ResNet32_baseline_distance_distance0.csv};

            \end{groupplot}
        \end{tikzpicture}
        \caption{Validation accuray and distance to the checkpoint for MobileNetV2 (upper) and ResNet32 (lower) trained without distance function (red) and with distance function (blue).}
    \end{center}
\end{figure}


[TODO: add gradient size to show that circular path]




\section{Distance function Hyperparameters}\label{res:Hyperparameters}
\subsection{strength}
Recall section \ref{eq:Loss_strength}, where we added strength as a
hyperparameter, controlling the influence of the distance function. As the RBF
Kernel is bound between 0 and 1, the strength hyperparameter will increase that
bound between 0 and the strength value.

For higher strength values, the validation accuracy receives an initial drop
after a checkpiont is added, which is larger the higher the strength parameter
value, see figure \ref{fig:Results_strength}. This may be due to the increased
influence of the strength function. Each distance between the parameters and the
checkpoint starts at 0, therefore the values of the exponential function starts
at $strength \cdot 1$. A higher strength value will lead to a higher influence
on the loss function, and consequently on the gradient. The gradient of the
cross-entropy loss will become insignificant, and the optimizer will update the
weights without regards to the validation accuracy, hence the drop at the
beginning. 

The distance plot reflects this behaviour, as we can see an larger increase in
distance for larger strength values. After the initial step however, the
distance quickly converges. This is due to the fact that after the initial
strong decrease, the value of the distance function quickly comes close to 0.
Therefore, the distance Kernel becomes insignificant again, and the
Cross-Entropy loss is followed. The validation accuracy reflects this behaviour,
as after the initial drop the accuracy recovers to the level of before. This
provides additional insight in the loss landscape. As the network distances from
it`s current position on the loss landscpae, but is still able to reach high
accuracy, there have to be areas of high accuracy everywhere on the landscape.
This is in consonance to other research in this field, as discussed in section
\ref{loss_landscape}.

As a result, the value of the strength doesn't matter in long term at least. It
merely increases the distance to the checkpoint by defining how long the
distance term is important to the loss function. But after that, the normal loss
is followed again. As the last section and literature suggests, area of low loss
and high accuray can be found everywhere. Therefore, the same accuracy as before
can be reached, no matter the value of the strength. On the other hand, it is
also unlikely that it will outperform the last best value, as new areas are not
generally better than other. That's why the effect of the distance doesn't
transfer to the validation accuracy.



\begin{figure}[h]\label{fig:Results_strength}
    \begin{center}
        \begin{tikzpicture}
            \begin{groupplot}[
                group style={
                group size=2 by 2,
                horizontal sep=10pt,
                vertical sep=10pt,
                group name=G},
                width=8cm,
                restrict x to domain=0:200
            ]

            \nextgroupplot[
            grid=major, 
            grid style={dashed,gray!30},
            % xlabel=Epoch,
            y label style={at={(axis description cs:0.1,0)},anchor=south},
            ylabel=Validation Accuracy,
            xticklabels={,,},
            ymin=0.6,
            legend pos= south east,
            xmin=-10]
            \addplot[mark=None, color=red] 
                table[x=Step, y=Value, col sep=comma]{images/network_csv/strength/MobileNetV2/MobileNetV2_strength_e1_validation_acuracy.csv};
            \addlegendentry{$s=0.1$}
            \addplot[mark=None, color=blue] 
                table[x=Step, y=Value, col sep=comma]{images/network_csv/baseline/MobileNetV2/MobileNetV2_baseline_distance_validation_acuracy.csv};
            \addlegendentry{$s=1$}
            \addplot[mark=None, color=orange] 
                table[x=Step, y=Value, col sep=comma]{images/network_csv/strength/MobileNetV2/MobileNetV2_strength_e2_validation_acuracy.csv};
            \addlegendentry{$s=10$}
            \addplot[mark=None, color=green] 
                table[x=Step, y=Value, col sep=comma]{images/network_csv/strength/MobileNetV2/MobileNetV2_strength_e3_validation_acuracy.csv};
            \addlegendentry{$s=100$}

            \nextgroupplot[
                grid=major, 
                grid style={dashed,gray!30},
                % xlabel=Epoch,
                y label style={at={(-0.1,0.5)},anchor=south},
                ylabel=Distance,
                yticklabel pos=right,
%                y label style={at={(axis description cs:0,0)},anchor=south},
                xticklabels={,,},
                ylabel near ticks]
            \addplot[mark=None, color=red] 
                table[x=Step, y=Value, col sep=comma]{images/network_csv/strength/MobileNetV2/MobileNetV2_strength_e1_distance0.csv};
            \addplot[mark=None, color=blue] 
                table[x=Step, y=Value, col sep=comma]{images/network_csv/baseline/MobileNetV2/MobileNetV2_baseline_distance_distance0.csv};
            \addplot[mark=None, color=orange] 
                table[x=Step, y=Value, col sep=comma]{images/network_csv/strength/MobileNetV2/MobileNetV2_strength_e2_distance0.csv};
            \addplot[mark=None, color=green] 
                table[x=Step, y=Value, col sep=comma]{images/network_csv/strength/MobileNetV2/MobileNetV2_strength_e3_distance0.csv};
    

            \nextgroupplot[
            grid=major, 
            grid style={dashed,gray!30},
            % x label style={at={(axis description cs:1.5,0)},anchor=west},
            xlabel=Epoch, % Set the labels
            % ylabel=Validation Accuracy,
            ymin=0.6,
            xmin=-10]
            \addplot[mark=None, color=red] 
                table[x=Step, y=Value, col sep=comma]{images/network_csv/strength/ResNet32/ResNet32_strength_e1_validation_acuracy.csv};
            \addplot[mark=None, color=blue] 
                table[x=Step, y=Value, col sep=comma]{images/network_csv/baseline/ResNet32/ResNet32_baseline_distance_validation_acuracy.csv};
            \addplot[mark=None, color=orange] 
                table[x=Step, y=Value, col sep=comma]{images/network_csv/strength/ResNet32/ResNet32_strength_e2_validation_acuracy.csv};
            \addplot[mark=None, color=green] 
                table[x=Step, y=Value, col sep=comma]{images/network_csv/strength/ResNet32/ResNet32_strength_e3_validation_acuracy.csv};
            
            \nextgroupplot[
                grid=major, 
                grid style={dashed,gray!30},
                %xlabel=Epoch, % Set the labels
                %ylabel=Distance,
                yticklabel pos=right,
                ylabel near ticks]
            \addplot[mark=None, color=red] 
                table[x=Step, y=Value, col sep=comma]{images/network_csv/strength/ResNet32/ResNet32_strength_e1_distance0.csv};
            \addplot[mark=None, color=blue] 
                table[x=Step, y=Value, col sep=comma]{images/network_csv/baseline/ResNet32/ResNet32_baseline_distance_distance0.csv};
            \addplot[mark=None, color=orange] 
                table[x=Step, y=Value, col sep=comma]{images/network_csv/strength/ResNet32/ResNet32_strength_e2_distance0.csv};
            \addplot[mark=None, color=green] 
                table[x=Step, y=Value, col sep=comma]{images/network_csv/strength/ResNet32/ResNet32_strength_e3_distance0.csv};

            \end{groupplot}
        \end{tikzpicture}
        \caption{Validation accuracy and Distance for MobileNetV2 (upper) and ResNet32 (lower) trained without distance function and different strength values.}
    \end{center}
\end{figure}

\subsection{width}
We have seen in chapter \ref{distance_function}, how the width $\sigma$
influences the distance function: The larger $\sigma$ gets, the wider it
becomes. Therefore, a larger $\sigma$ should result in the network distancing
further from it's checkpoint than for smaller values. That's exactly what can be
seen in figure \ref{fig:Results_width}: For an $\sigma^2$ of 0.1, the distance
follows the one of network trained without distance function, due to the
distance function quickly becoming 0 and therefore having no influence. The
larger the width, the slower the values of the distance function decrease for further
distance. Therefore, the gradient of the distance function will stay quite
large, pushing the network further away. This can be seen for higher $\sigma$
values, where the distance increases further.


Initially however, the distance plot doesn't reflect this behaviour. The curve
for $\sigma = 0.01$ has a higher distance value than for $\sigma = 0.001$.
That's because the gradient of the distance function is higher for small distances, the
smaller $\sigma$. But for larger distances, this effect flips with the gradient
of the larger width staying higher for more epochs. Therefore, the curves
intersect after additional epochs. The smaller width converges, while the larger
stays constant until eventually reaching it`s area of convergence. The
intersection happens later the larger the width, for $\sigma = 0.0001$, only
after 207 epochs.

This behaviour however has no influence on the validation accuracy, which stays
the same for every width, in contrast to the strength. This may be due to the
fact, that a larger width does in fact decreases the size of the gradient. A
smaller gradient over more epochs will be added to the gradient of the
Cross-Entropy Loss. The optimizer therefore nearly keeps on following the normal
gradient and descending the valley of low loss, with a small but steady push to
increase the distance to the checkpoint. Therefore, even if the network is
allowed to move quite freely on the loss surface, it finds areas of low loss and
high accuracy in distant areas of the checkpiont, meaning there has to be good
local minima everywhere on the landscape.

In particular, $\sigma^2 = 100$ seems to increase the distance far enough in a
small number of epochs. That's why this configuration is used as standard in the
following work.

\begin{figure}[h]\label{fig:Results_width}
    \begin{center}
        \begin{tikzpicture}
            \begin{groupplot}[
                group style={
                group size=2 by 2,
                horizontal sep=10pt,
                vertical sep=10pt,
                group name=G},
                width=8cm,
                restrict x to domain=0:400
            ]

            \nextgroupplot[
            grid=major, 
            grid style={dashed,gray!30},
            % xlabel=Epoch,
            ylabel=Validation Accuracy,
            xticklabels={,,},
            ymin=0.7,
            legend pos = south east,
            xmin=-10]
            \addplot[mark=None, color=red] 
                table[x=Step, y=Value, col sep=comma]{images/network_csv/width/MobileNetV2/MobileNetV2_width_e1_validation_acuracy.csv};
            \addlegendentry{$\sigma^2=1$}
            \addplot[mark=None, color=orange] 
                table[x=Step, y=Value, col sep=comma]{images/network_csv/width/MobileNetV2/MobileNetV2_width_e2_validation_acuracy.csv};
            \addlegendentry{$\sigma^2=10$}
            \addplot[mark=None, color=blue]
                table[x=Step, y=Value, col sep=comma]{images/network_csv/baseline/MobileNetV2/MobileNetV2_baseline_distance_validation_acuracy.csv};
            \addlegendentry{$\sigma^2=100$}
            \addplot[mark=None, color=green] 
                table[x=Step, y=Value, col sep=comma]{images/network_csv/width/MobileNetV2/MobileNetV2_width_e4_validation_acuracy.csv};
            \addlegendentry{$\sigma^2=1000$}

            \nextgroupplot[
                grid=major, 
                grid style={dashed,gray!30},
                % xlabel=Epoch,
                ylabel=Distance,
                yticklabel pos=right,
                xticklabels={,,},
                ylabel near ticks]
            \addplot[mark=None, color=red] 
                table[x=Step, y=Value, col sep=comma]{images/network_csv/width/MobileNetV2/MobileNetV2_width_e1_distance0.csv};
            \addplot[mark=None, color=orange] 
                table[x=Step, y=Value, col sep=comma]{images/network_csv/width/MobileNetV2/MobileNetV2_width_e2_distance0.csv};
            \addplot[mark=None, color=blue] 
                table[x=Step, y=Value, col sep=comma]{images/network_csv/baseline/MobileNetV2/MobileNetV2_baseline_distance_distance0.csv};
            \addplot[mark=None, color=green] 
                table[x=Step, y=Value, col sep=comma]{images/network_csv/width/MobileNetV2/MobileNetV2_width_e4_distance0.csv};
    

            \nextgroupplot[
            grid=major, 
            grid style={dashed,gray!30},
            xlabel=Epoch, % Set the labels
            ylabel=Validation Accuracy,
            ymin=0.7,
            xmin=-10]
            \addplot[mark=None, color=red] 
                table[x=Step, y=Value, col sep=comma]{images/network_csv/width/ResNet32/ResNet32_width_e1_validation_acuracy.csv};
            \addplot[mark=None, color=orange] 
                table[x=Step, y=Value, col sep=comma]{images/network_csv/width/ResNet32/ResNet32_width_e2_validation_acuracy.csv};
            \addplot[mark=None, color=blue] 
                table[x=Step, y=Value, col sep=comma]{images/network_csv/baseline/ResNet32/ResNet32_baseline_distance_validation_acuracy.csv};
            \addplot[mark=None, color=green] 
                table[x=Step, y=Value, col sep=comma]{images/network_csv/width/ResNet32/ResNet32_width_e4_validation_acuracy.csv};

            
            \nextgroupplot[
                grid=major, 
                grid style={dashed,gray!30},
                xlabel=Epoch, % Set the labels
                ylabel=Distance,
                yticklabel pos=right,
                ylabel near ticks]
            \addplot[mark=None, color=red] 
                table[x=Step, y=Value, col sep=comma]{images/network_csv/width/ResNet32/ResNet32_width_e1_distance0.csv};
            \addplot[mark=None, color=orange] 
                table[x=Step, y=Value, col sep=comma]{images/network_csv/width/ResNet32/ResNet32_width_e2_distance0.csv};
            \addplot[mark=None, color=blue] 
                table[x=Step, y=Value, col sep=comma]{images/network_csv/baseline/ResNet32/ResNet32_baseline_distance_distance0.csv};
            \addplot[mark=None, color=green] 
                table[x=Step, y=Value, col sep=comma]{images/network_csv/width/ResNet32/ResNet32_width_e4_distance0.csv};

            \end{groupplot}
        \end{tikzpicture}
        \caption{Validation accuray and Distance for MobileNetV2 (upper) and ResNet32 (lower) trained with different widths of the distance function.}
    \end{center}
\end{figure}



\section{Multiple Checkpoint}\label{res:Multiple} In chapter \ref{cha:Methods},
we have seen that we can not only incorporate one but also multiple checkpoints
in the loss function, on which this section will focus. We also try to not only
add these checkpoints as seperate terms, but rather incorporate them in one
larger checkpoint. 

\subsection{multiple}
We add a checkpoint after every 150 epochs that are performed, resulting in a
total of 3 checkpoints after 600 epochs.

For the validation accuracy, we can see the same pattern as in section
\ref{res:Hyperparameters}. For low strength, we experience no effect at all. If
we increase the strength there is an initial drop in the accuracy. For
MobileNetV2, the accuracy recovers and even tops the baseline, with a small but
increasing margin for every new checkpoints that is added, coming to an
improvement of 0.8\% over the baseline after 600 epochs. In contrast, ResNet32
doesn't seem to benefit from multiple checkpoints, the accuracy even decreases
for larger strength values.

The pattern of MobileNetV2 is quite similar to warm restarts, where for every
restart the reached accuracy increases. For a in depth comparison, see section
\ref{res:Learning rate}.

\begin{figure}[h]\label{fig:Results_multiple}
    \begin{center}
        \begin{tikzpicture}
            \begin{groupplot}[
                group style={
                group size=2 by 1,
                horizontal sep=10pt,
                vertical sep=10pt,
                group name=G},
                width=8cm
            ]

            \nextgroupplot[
            grid=major, 
            grid style={dashed,gray!30},
            xlabel=Epoch,
            ylabel=Validation Accuracy,
            ymin=0.6,
            ymax=1,
            legend pos = south east,
            xmin=-10]
            \addplot[mark=None, color=red] 
                table[x=Step, y=Value, col sep=comma]{images/network_csv/multiple/MobileNetV2/MobileNetV2_multiple_validation_acuracy.csv};
                \addlegendentry{$s=1$}
            \addplot[mark=None, color=green] 
                table[x=Step, y=Value, col sep=comma]{images/network_csv/multiple/MobileNetV2/MobileNetV2_multiple_f10_validation_acuracy.csv};
                \addlegendentry{$s=10$}
            \addplot[mark=None, color=blue] 
                table[x=Step, y=Value, col sep=comma]{images/network_csv/multiple/MobileNetV2/MobileNetV2_multiple_f100_validation_acuracy.csv};
                \addlegendentry{$s=100$}
            
    

            \nextgroupplot[
            grid=major, 
            grid style={dashed,gray!30},
            xlabel=Epoch, % Set the labels
            ytick pos=right,
            ymin=0.6,
            ymax=1,
            xmin=-10]
            \addplot[mark=None, color=red] 
                table[x=Step, y=Value, col sep=comma]{images/network_csv/multiple/ResNet32/ResNet32_multiple_validation_acuracy.csv};
            \addplot[mark=None, color=green] 
                table[x=Step, y=Value, col sep=comma]{images/network_csv/multiple/ResNet32/ResNet32_multiple_f10_validation_acuracy.csv};
            \addplot[mark=None, color=blue] 
                table[x=Step, y=Value, col sep=comma]{images/network_csv/multiple/ResNet32/ResNet32_multiple_f100_validation_acuracy.csv};
            % [TODO: add ResNet Multiple]

            \end{groupplot}
        \end{tikzpicture}
        \caption{Validation accuray and Distance for MobileNetV2 (upper) and ResNet32 (lower) trained with multiple checkpoints. [TODO: add for ResNet]}
    \end{center}
\end{figure}

The influence on the distance to the checkpoints is more interesting however. In
general, an additional checkpoint seems to decrease the distance to the others.
This is quite counterintuitive at first glance. Suppose for one parameter, we
have checkpoint value $a$ and create a new one at our current value $b<a$. To
increase the distance to both, we would just have to walk in direction $c<b$. As
we have seen in previous section, this would be sufficient as areas of low error
exist everywhere. But why is the distance then decreased for our first
checkpoint? The reason might be the $L_2$ regularization, as it restricts the
weights to small sizes. At some point, it might be less costly to decrease the
weights again for a smaller $L_2$, traded off against a bit smaller distance to
other checkpoints. This restriction in weight space is probably why new terms
lead to a decrease of the distance to existing checkpoints. A network trained
without $L_2$ regularization futher supports this explanation. Here, the
distance increases further for each new checkpoint that is added. [add figure??]

\begin{figure}[h]\label{fig:Results_multiple_distance}
    \begin{center}
        \begin{tikzpicture}
            \begin{groupplot}[
                group style={
                group size=3 by 2,
                horizontal sep=10pt,
                vertical sep=10pt,
                group name=G},
                width=5cm
            ]

            \nextgroupplot[
            title=checkpoint 1,
            grid=major, 
            grid style={dashed,gray!30},
            %x label style={at={(axis description cs:1.5,0)},anchor=north},
            % y label style={at={(axis description cs:-0.1,.5)},rotate=90,anchor=south},
            %xlabel=Epoch,
            ylabel=distance,
            ylabel near ticks,
            ymax=8000,
            xmin=140,
            xticklabels={,,}]
            \addplot[mark=None, color=red] 
                table[x=Step, y=Value, col sep=comma]{images/network_csv/multiple/MobileNetV2/MobileNetV2_multiple_distance0.csv};
            \addplot[mark=None, color=green] 
                table[x=Step, y=Value, col sep=comma]{images/network_csv/multiple/MobileNetV2/MobileNetV2_multiple_f10_distance0.csv};
            \addplot[mark=None, color=blue] 
                table[x=Step, y=Value, col sep=comma]{images/network_csv/multiple/MobileNetV2/MobileNetV2_multiple_f100_distance0.csv};

            \nextgroupplot[
            title=checkpoint 2,
            grid=major, 
            grid style={dashed,gray!30},
            yticklabels={,,}
            ymax=8000,
            xmin=290,
            xticklabels={,,}]
            \addplot[mark=None, color=red] 
                table[x=Step, y=Value, col sep=comma]{images/network_csv/multiple/MobileNetV2/MobileNetV2_multiple_distance1.csv};
            \addplot[mark=None, color=green] 
                table[x=Step, y=Value, col sep=comma]{images/network_csv/multiple/MobileNetV2/MobileNetV2_multiple_f10_distance1.csv};
            \addplot[mark=None, color=blue] 
                table[x=Step, y=Value, col sep=comma]{images/network_csv/multiple/MobileNetV2/MobileNetV2_multiple_f100_distance1.csv};

            \nextgroupplot[
            title=checkpoint 3,
            grid=major, 
            grid style={dashed,gray!30},
            yticklabels={,,},
            ymax=8000,
            legend pos = outer north east,
            xmin=440,
            xticklabels={,,}]
            \addplot[mark=None, color=red] 
                table[x=Step, y=Value, col sep=comma]{images/network_csv/multiple/MobileNetV2/MobileNetV2_multiple_distance2.csv};
                \addlegendentry{$s=1$}
            \addplot[mark=None, color=green] 
                table[x=Step, y=Value, col sep=comma]{images/network_csv/multiple/MobileNetV2/MobileNetV2_multiple_f10_distance2.csv};
                \addlegendentry{$s=10$}
            \addplot[mark=None, color=blue] 
                table[x=Step, y=Value, col sep=comma]{images/network_csv/multiple/MobileNetV2/MobileNetV2_multiple_f100_distance2.csv};
                \addlegendentry{$s=100$}

%--------------------ResNet----------------------------------------------------

            \nextgroupplot[
            grid=major, 
            grid style={dashed,gray!30},
            ylabel=distance,
            ylabel near ticks,
            ymax=8000,
            xmin=140]
            \addplot[mark=None, color=red] 
                table[x=Step, y=Value, col sep=comma]{images/network_csv/multiple/ResNet32/ResNet32_multiple_distance0.csv};
            \addplot[mark=None, color=green] 
                table[x=Step, y=Value, col sep=comma]{images/network_csv/multiple/ResNet32/ResNet32_multiple_f10_distance0.csv};
            \addplot[mark=None, color=blue] 
                table[x=Step, y=Value, col sep=comma]{images/network_csv/multiple/ResNet32/ResNet32_multiple_f100_distance0.csv};

            \nextgroupplot[
            grid=major, 
            grid style={dashed,gray!30},
            xlabel=Epoch,
            yticklabels={,,}
            ymax=8000,
            xmin=290]
            \addplot[mark=None, color=red] 
                table[x=Step, y=Value, col sep=comma]{images/network_csv/multiple/ResNet32/ResNet32_multiple_distance1.csv};
            \addplot[mark=None, color=green] 
                table[x=Step, y=Value, col sep=comma]{images/network_csv/multiple/ResNet32/ResNet32_multiple_f10_distance1.csv};
            \addplot[mark=None, color=blue] 
                table[x=Step, y=Value, col sep=comma]{images/network_csv/multiple/ResNet32/ResNet32_multiple_f100_distance1.csv};

            \nextgroupplot[
            grid=major, 
            grid style={dashed,gray!30},
            yticklabels={,,},
            ymax=8000,
            xmin=440]
            \addplot[mark=None, color=red] 
                table[x=Step, y=Value, col sep=comma]{images/network_csv/multiple/ResNet32/ResNet32_multiple_distance2.csv};
            \addplot[mark=None, color=green] 
                table[x=Step, y=Value, col sep=comma]{images/network_csv/multiple/ResNet32/ResNet32_multiple_f10_distance2.csv};
            \addplot[mark=None, color=blue] 
                table[x=Step, y=Value, col sep=comma]{images/network_csv/multiple/ResNet32/ResNet32_multiple_f100_distance2.csv};

            \end{groupplot}
        \end{tikzpicture}
        \caption{Distance plots for MobileNetV2 (upper) and ResNet32 (lower) trained with multiple checkpoints and different strength values. [TODO: add for ResNet]}
    \end{center}
\end{figure}

In short term however, new checkpoint produce quite different influence on the
distance to existing ones. For a strength of $s=1$ we can see a slight decrease
which turns into a longer increase until the next checkpoint. For $s=10$, we see
an aprupt decrease with an small peak after. For $s=100$, there only is a peak
at the beginning which quickly diminishes. Noticeable, this pattern repeats for
the other checkpoints and therefore seems unlikely to be a coincidence. The size
of these hills can be tracked to the size of the strength. Larger strength
introduces a larger gradient and therefore step size at the beginning, making
the hills larger. This doesn't explain the different orientation of these hills.
For the case of $s=100$, the initial increase of distance seems reasonable. As
the value of the old distance function is not 0 when a new checkpoints is
created, the current gradient should face in direction away from the last
checkpoint. A new checkpoint term in the loss function introduces a gradient
boost, as section \ref{sub:Effect_on_Gradient} discussed. Because momentum
preserves the old gradient, it will acclerate in direction away from the first
and the second checkpoint, therefore leading to a hill. However by this
explanation, the pattern should repeat for the other strength values, which is
not the case. Therefore, the true cause of the short term effects remains
unclear.


- gradient size becomes smaller for distance term -> maybe hint that SGD wants
to stay in that area and distance term tries to push out -> gradient direction
would be interesting (but too far for this work)

\subsection{merge}
A slightly different approach is to merge the checkpoints, described in section
\ref{sub:Multiple_checkpoints}.
[TODO: add more networks for merge]










\section{lr}\label{res:Learning_rate}
\subsection{Scheduler}\label{res:Scheduler}
Learning rate Scheduler were introduced in section \ref{sub:Learing_rate_decay},
where the learning rate was not held constant, but decayed over time. One
further addition were warm restarts, where after the decay we set the learning
rate back up to the inital and repeat this cycle several times. We use a step
decay scheduler with $\gamma = 0.1$ for every 50 epochs, and a warm restart
after 150 epochs. For cosine decay with warm restart, we set $\epsilon_{min}=0$,
$\epsilon_{max}=0.001$ and $T_0=150$.


For the validation accuracy, we can see that schedulers outperform a fixed
learning rate. At the beginning of every cycle, the performance drops due to
high learning rate. But when the learning rate decays again, the accuracy
recovers. Furthermore, the maximum accuracy of every cycle slightly increases,
until eventually coming to convergence. This leads to a maximum accuracy of
around 95\% for both ResNet32 and MobileNetV2.


\begin{figure}[h]\label{fig:Results_scheduler}
    \begin{center}
        \begin{tikzpicture}
            \begin{groupplot}[
                group style={
                group size=2 by 2,
                horizontal sep=10pt,
                vertical sep=10pt,
                group name=G},
                width=8cm
            ]

            \nextgroupplot[
            grid=major, 
            grid style={dashed,gray!30},
            % xlabel=Epoch,
            ylabel=Validation Accuracy,
            xticklabels={,,},
            ymin=0.75,
            legend pos = south east,
            xmin=-10]
            \addplot[mark=None, color=red] 
                table[x=Step, y=Value, col sep=comma]{images/network_csv/baseline/MobileNetV2/MobileNetV2_baseline_validation_acuracy.csv};
                \addlegendentry{fixed lr}
            \addplot[mark=None, color=green] 
                table[x=Step, y=Value, col sep=comma]{images/network_csv/scheduler/MobileNetV2/MobileNetV2_scheduler_step_validation_acuracy.csv};
                \addlegendentry{step decay}
            \addplot[mark=None, color=blue] 
                table[x=Step, y=Value, col sep=comma]{images/network_csv/scheduler/MobileNetV2/MobileNetV2_scheduler_cosine_validation_acuracy.csv};
                \addlegendentry{cosine decay}
            \nextgroupplot[
                grid=major, 
                grid style={dashed,gray!30},
                % xlabel=Epoch,
                ylabel=Distance,
                yticklabel pos=right,
                xticklabels={,,},
                ylabel near ticks]
            \addplot[mark=None, color=red] 
                table[x=Step, y=Value, col sep=comma]{images/network_csv/baseline/MobileNetV2/MobileNetV2_baseline_distance0.csv};
            \addplot[mark=None, color=green] 
                table[x=Step, y=Value, col sep=comma]{images/network_csv/scheduler/MobileNetV2/MobileNetV2_scheduler_step_distance0.csv};
            \addplot[mark=None, color=blue] 
                table[x=Step, y=Value, col sep=comma]{images/network_csv/scheduler/MobileNetV2/MobileNetV2_scheduler_cosine_distance0.csv};
    

            \nextgroupplot[
            grid=major, 
            grid style={dashed,gray!30},
            xlabel=Epoch, % Set the labels
            ylabel=Validation Accuracy,
            ymin=0.7,
            xmin=-10]
            \addplot[mark=None, color=red] 
                table[x=Step, y=Value, col sep=comma]{images/network_csv/baseline/ResNet32/ResNet32_baseline_validation_acuracy.csv};
            %\addplot[mark=None, color=green] 
             %   table[x=Step, y=Value, col sep=comma]{images/network_csv/scheduler/ResNet32/ResNet32_scheduler_step_validation_acuracy.csv};
            \addplot[mark=None, color=blue] 
                table[x=Step, y=Value, col sep=comma]{images/network_csv/scheduler/ResNet32/ResNet32_scheduler_cosine_validation_acuracy.csv};


            
            \nextgroupplot[
                grid=major, 
                grid style={dashed,gray!30},
                xlabel=Epoch, % Set the labels
                ylabel=Distance,
                yticklabel pos=right,
                ylabel near ticks]
            \addplot[mark=None, color=red] 
                table[x=Step, y=Value, col sep=comma]{images/network_csv/baseline/ResNet32/ResNet32_baseline_distance0.csv};
%           \addplot[mark=None, color=green] 
%                table[x=Step, y=Value, col sep=comma]{images/network_csv/scheduler/ResNet32/ResNet32_scheduler_step_distance0.csv};
            \addplot[mark=None, color=blue] 
                table[x=Step, y=Value, col sep=comma]{images/network_csv/scheduler/ResNet32/ResNet32_scheduler_cosine_distance0.csv};
            \end{groupplot}
        \end{tikzpicture}
        \caption{Validation accuray and Distance for MobileNetV2 (upper) and ResNet32 (lower) trained with learning rate schedulers.}
    \end{center}
\end{figure}


The distance plot now also looks a bit different compared to a fixed lr. We can
see that after an initial increase in distance to the checkpoint, the distance
reaches a local maximum and then actually decreases again until coming into an
area of convergence. For each restart performed, the same pattern repeats again.
For step decay, we can see hard edges in contrast to smooth curves of cosine
decay. The similarity to the alteration of the learning rate gives evidence that
the decrease of distance is caused by the learning rate. 

One possible explanation is that the decrease in learnig rate could lead to
smaller increase in distance by the pure fact, that smaller learning rate leads
to smaller updates of the weights and threfeore smaller increase in distance.
However, this couldn`t explain why the distance itself actually keeps
decreasing, not just smaller increasing. Only could try to explain this with the
$L_2$ Loss and a similar argument to section \ref{res:Multiple}. For high
learning rate, a larger Momentum builds up. This leads to parameter values,
which are very larger. After the momentum is slowed down, the $L_2$
regularization keeps decreasing the weights again until reaching an equilibrium.
But larger convergences in the following cycles make this explanation doubtful. 

\begin{comment}
Another idea could relate to the wedges. If we enter a wedge, it might be
possible that we end up on a circular path around the checkpoint. This might
lead to a valley which decreases the distance again. For large learning rate, we
can just jump between those valley, but for small learnig rates, we just follow
these valleys.
\end{comment}

We have seen that whenever a warm restart is performed, the accuracy drops.
However with decreasing learning rate, the network recovers and even tops the
performance of the previous cycle. This pattern is quite similar to effect of
multiple checkpoint of section \ref{res:Multiple}. Section
\ref{sub:Effect_on_Gradient} further motivates a comparison: We suggested that a
checkpoint would initially just increase the size of the gradient, rather than
changing its orientation, given that the orientation is stable over some epochs.
Therefore, the same effect on the parameter update, but rather from a larger
gradient itself than a larger learning rate, should be achieved.

Figure \ref{fig:Cosine_Multiple} shows a comparison. As mentioned, the pattern
is quite similar, with a drop in validation accuracy after the warm restart or
checkpoint. The top accuracy in each cycle also tops the last one for both
cases. The effect is similar with around 1\% against 0.8\% increase in maximum accuracy
between the first and last cycle for cosine decay stronger than for multiple
checkpoints. Additionallly, the network with cosine decay gets a much better absolute
accuracy. That's probably due to the small learning rate at the end of each
cycle, where the network can exploit a small region to fine adjust the parameter
values.

\begin{figure}[h]\label{fig:Cosine_Multiple}
    \begin{center}
        \begin{tikzpicture}
            \begin{axis}[
                grid=major, 
                grid style={dashed,gray!30},
                xlabel=Epoch,
                ylabel=Validation Accuracy,
                ymin=0.75,
                legend pos = outer north east,
                xmin=-10,
                width=10cm]
                
                \addplot[mark=None, color=red] 
                    table[x=Step, y=Value, col sep=comma]{images/network_csv/scheduler/MobileNetV2/MobileNetV2_scheduler_cosine_validation_acuracy.csv};
                    \addlegendentry{cosine decay}
                \addplot[mark=None, color=green] 
                    table[x=Step, y=Value, col sep=comma]{images/network_csv/multiple/MobileNetV2/MobileNetV2_multiple_f100_validation_acuracy.csv};
                    \addlegendentry{multiple checkpoints}
                
            \end{axis}            
        \end{tikzpicture}
        \caption{Validation accuray and Distance for MobileNetV2 (upper) and ResNet32 (lower) trained with learning rate schedulers. [TODO: add for ResNet]}
    \end{center}
\end{figure}

Would a combination of both techniques lead to an even better result?
[TODO:add results and other factor] 







\subsection{wrong lr}

We might suspect that a learning rate decay should be more robust to a
suboptimal inital learning rate. At first, this seems to be case. Where a
network without decay only reaches a accuracy of 60\% and then decreases, a
scheduler reaches a comparable accuracy to the best network. But if we perform a
warm restart, the maximum accuracy drops for each cycle, so the network gets
worse with increasing cycles. If we add the distance function however, we can
see that it stabilizes the training so that the maximum accuracy stays the same
for each cycle. This leads to a performance difference of 8\% for MobileNetV2 in
favour of the distance function after 600 epochs. The difference is also present
from the beginning of each restart and remains stable. However, we loose the
effect from above, that each warm restart boosts the performance. Nevertheless,
this is a significant improvement, which is stable across both ResNet32 and
MobileNetV2.

\begin{figure}[h]\label{fig:Results_wrong_lr}
    \begin{center}
        \begin{tikzpicture}
            \begin{groupplot}[
                group style={
                group size=2 by 2,
                horizontal sep=10pt,
                vertical sep=10pt,
                group name=G},
                width=8cm
            ]

            \nextgroupplot[
            grid=major, 
            grid style={dashed,gray!30},
            % xlabel=Epoch,
            ylabel=Validation Accuracy,
            xticklabels={,,},
            ymin=0.3,
            legend pos = south east,
            xmin=-10]
            \addplot[mark=None, color=red] 
                table[x=Step, y=Value, col sep=comma]{images/network_csv/scheduler/MobileNetV2/lr1/MobileNetV2_scheduler_cosine_lr1_validation_acuracy.csv};
            \addplot[mark=None, color=green] 
               table[x=Step, y=Value, col sep=comma]{images/network_csv/scheduler/MobileNetV2/lr1/MobileNetV2_scheduler_cosine_distance_lr1_validation_acuracy.csv};
            \addplot[mark=None, color=blue] 
                table[x=Step, y=Value, col sep=comma]{images/network_csv/lr/MobileNetV2/run-mobileNetV2_baseline_distance_lr1_1-tag-Validation_accuracy.csv};
            \addplot[mark=None, color=orange] 
                table[x=Step, y=Value, col sep=comma]{images/network_csv/lr/MobileNetV2/run-mobileNetV2_baseline_lr1_1-tag-Validation_accuracy.csv};
                % add for lr1 longer baseline
                

            \nextgroupplot[
                grid=major, 
                grid style={dashed,gray!30},
                % xlabel=Epoch,
                ylabel=Distance,
                yticklabel pos=right,
                xticklabels={,,},
                ylabel near ticks]
            \addplot[mark=None, color=red] 
                table[x=Step, y=Value, col sep=comma]{images/network_csv/scheduler/MobileNetV2/lr1/MobileNetV2_scheduler_cosine_distance_lr1_distance0.csv};
            \addplot[mark=None, color=green] 
                table[x=Step, y=Value, col sep=comma]{images/network_csv/scheduler/MobileNetV2/lr1/MobileNetV2_scheduler_cosine_lr1_distance0.csv};
            \addplot[mark=None, color=blue] 
                table[x=Step, y=Value, col sep=comma]{images/network_csv/lr/MobileNetV2/run-mobileNetV2_baseline_distance_lr1_1-tag-distance0.csv};
            \addplot[mark=None, color=orange] 
                table[x=Step, y=Value, col sep=comma]{images/network_csv/lr/MobileNetV2/run-mobileNetV2_baseline_lr1_1-tag-distance0.csv};
                % add for lr1 baseline

        

            \nextgroupplot[
            grid=major, 
            grid style={dashed,gray!30},
            xlabel=Epoch, % Set the labels
            ylabel=Validation Accuracy,
            ymin=0.5,
            xmin=-10]
            \addplot[mark=None, color=red] 
                table[x=Step, y=Value, col sep=comma]{images/network_csv/scheduler/ResNet32/lr1/ResNet32_scheduler_cosine_lr1_validation_acuracy.csv};
            \addplot[mark=None, color=green] 
                table[x=Step, y=Value, col sep=comma]{images/network_csv/scheduler/ResNet32/lr1/ResNet32_scheduler_cosine_lr1_distance_validation_acuracy.csv};
        

            
            \nextgroupplot[
                grid=major, 
                grid style={dashed,gray!30},
                xlabel=Epoch, % Set the labels
                ylabel=Distance,
                yticklabel pos=right,
                ylabel near ticks]
            \addplot[mark=None, color=red] 
                table[x=Step, y=Value, col sep=comma]{images/network_csv/scheduler/ResNet32/lr1/ResNet32_scheduler_cosine_lr1_distance0.csv};
            \addplot[mark=None, color=green] 
                table[x=Step, y=Value, col sep=comma]{images/network_csv/scheduler/ResNet32/lr1/ResNet32_scheduler_cosine_lr1_distance_distance0.csv};

            \end{groupplot}
        \end{tikzpicture}
        \caption{Validation accuray and Distance for MobileNetV2 (upper) and ResNet32 (lower) trained with learning rate schedulers. [TODO: add for ResNet]}
    \end{center}
\end{figure}

The effect can also be spotted if we compare without scheduler. Here, both
networks decrease in accuracy over time, but the distance term seems to reduce
that decrease. 




- show general improvement of scheduler
- step vs cosine
- combine both with distance show improvement
- compare to multiple as another method of gradient boost


- gradient:
    warm restart decreases size and increase again with restart
    but effect is reversed for wrong lr, maybe distance term leads to more consistent gradient size -> better performance
    but has to carefully look how large gradient size difference is 
    



result idea:
- difference with wrong lr
- too high doenst lead to convergence, too low leads to slower one
- if kernel is applied, starts distancing
- 
-
for scheduler:
- maybe sgd walks in area of low accuracy, cannot recover (what is this area)
- distance kernel helps escape from this area



\section{Computational Cost}\label{Res:Computational_cost}

Improvement in performance often arises with an increase in computational cost.
In Section \ref{sub:Computational_Analysis} this was discussed in a theoretical
setting, suggesting that the additional cost would scale linear with each
checkpoint. If we analyze the baseline network without distance from section
\ref{res:Baseline}, we reach a per epoch training time of 68s, trained on the
TCML Cluster [Add reference]. For the network trained with distance function, we
start with the same epoch time, as the analysis expected. That`s due to the
fact, that we also start training without the distance function. But once we add
a checkpoint, the epoch time rises by 21s to a total of 89s, see figure
\ref{fig:Epoch_time}. If we add multiple checkpoints, the increase for each
stays constant at around 21ms. Therefore the cost indeed scales linear to the
number of checkpoints.

\begin{figure}[h]\label{fig:Epoch_time}
    \begin{center}
        \begin{tikzpicture}
            \begin{groupplot}[
                group style={
                group size=2 by 1,
                horizontal sep=10pt,
                group name=G},
                width=8cm,
            ]
                
            \nextgroupplot[
            grid=major, 
            grid style={dashed,gray!30},
            xlabel=Epoch, % Set the labels
            ylabel=Epoch time,
            ylabel near ticks,
            legend pos=north west,
            legend style={nodes={scale=0.7, transform shape}}
            ]
            \addplot[mark=None, color=red] 
                table[x=Step, y=Value, col sep=comma]{images/network_csv/baseline/MobileNetV2/MobileNetV2_baseline_epoch_time.csv};
                \addlegendentry{baseline}
            \addplot[mark=None, color=green] 
                table[x=Step, y=Value, col sep=comma]{images/network_csv/baseline/MobileNetV2/MobileNetV2_baseline_distance_epoch_time.csv};
                \addlegendentry{one checkpoint}
            \addplot[mark=None, color=blue] 
                table[x=Step, y=Value, col sep=comma]{images/network_csv/multiple/MobileNetV2/MobileNetV2_multiple_epoch_time.csv};
                \addlegendentry{multiple checkpoints}
            
            \nextgroupplot[
            grid=major, 
            grid style={dashed,gray!30},
            xlabel=Epoch, % Set the labels
            ytick pos=right
            ]
            \addplot[mark=None, color=red] 
                table[x=Step, y=Value, col sep=comma]{images/network_csv/baseline/ResNet32/ResNet32_baseline_epoch_time.csv};
            \addplot[mark=None, color=green] 
                table[x=Step, y=Value, col sep=comma]{images/network_csv/baseline/ResNet32/ResNet32_baseline_distance_epoch_time.csv};
            \addplot[mark=None, color=blue] 
                table[x=Step, y=Value, col sep=comma]{images/network_csv/multiple/ResNet32/ResNet32_multiple_epoch_time.csv};
            

            \end{groupplot}
            
        \end{tikzpicture}
        \caption{Epoch time in seconds for different configurations of MobileNetV2.}
    \end{center}
\end{figure}

\section{ensemble methods}\label{res:Ensemble}
Ensemble methods were introduced in section \ref{sub:Ensemble_Methods}, the idea
beeing that the combined prediction of multiple networks would result in better
performance. However, the networks need uncorellated errors for that
increasement.

In the approach from \cite{loshchilov2016sgdr}, they took a snapshot of the
network at the end of each cycle. This procedure results in a performance boost
of around 1\% against a standard consine decay from section \ref{res:Learning rate},
without any additional training cost. The high learning rate after the warm
restart ensures that the each new snapshot distances itself in the parameter
space from the old. The hope is, that this distance results in uncorellated
errors.

As motivated in section \ref{sub:Approach}, we try to increase that distance
more explicitly with our distance function. We use the same network as in
section \ref{res:Multiple} and take new snapshots before adding a new
checkpoint. After taking the snapshot, we add it to the ensemble, which consists
of all previous snaphsots and the model that is currently trained on. For the
ensemble prediction, we use the model averaging technique from
\ref{sub:Ensemble_methods}.


This leads to a ensemble accuracy of 94\%, which is 4\% better than
the validation accuracy of the single network. Figure [add figure] shows a
stepwise increase in ensemble accuracy after 150, 300, 450 and 600 epochs, when
a snapshot is added to the ensemble. This shows that each new network adds
performance to the ensemble. However, the network doesn´t reach the performance
of cosine decay with warm restart and is even similar to a standard network
without any decay or distance function.

The difference is probably due to the fact, that cosine decay reaches a better
performance for each snapshots of the ensemble. The predictions may therefore be
better in overall, even if they are not more versatile.

To see if a ensemble would benefit from snapshots that have a further distance
between each other and therefore have a lower corellation, we increase
thestrength parameter. However, the ensemble accuracy doesn't benefit from this.
Instead, the accuracy gets even worse for $s=10$ and $s=100$. 

In order to check if further distance truly means more different predictions, we
compare the individual predictions of the networks. Table [add reference] shows
the results. If we compare the different strength values of the distance
function, we can see that the number of examples for which the predictions of
the networks is the same has no difference between the strength values. This
suggests that although the snapshots of the ensembles differ in distance to each
other, this doens't result in different predictions. Even compared to the
baseline, the coherence seems similar. Another possible explanation could be the
chracteristics of the training set, which may contain some easy and some
difficult examples. As a fixed learning rate doesn't exploit the landscape
enough, all networks may get the same easy examples right, regardless of their
position on the loss landscape. But for hard training examples, all these
networks fail.



For cosine decay, the coherence is even higher. This seems reasonable, as
snaphsots which perform better necessarly need to have more similar predictions,
as there are less options where they are wrong and can therefore disagree.
Nevertheless it is remarkable that although the distance between the networks
for cosine decay and a fixed learning rate is quite similar, the prediction
coherence differs strongly. This further suggests that a distance in parameter
space doens't result in different behaviour. Cosine decay probably exploits the
local landscape better due to a low learning rate and therefore arrives at more
similar snapshots.




\begin{comment}
example picture:
\begin{figure}[h]\label{fig:MobileNetV2_baseline}
    \begin{center}
        \begin{tikzpicture}
            \begin{groupplot}[
                group style={
                group size=2 by 1,
                horizontal sep=10pt,
                group name=G},
                width=8cm,
            ]

            \nextgroupplot[
            grid=major, 
            grid style={dashed,gray!30},
            xlabel=Epoch, % Set the labels
            ylabel=Validation Accuracy,
            ymin=0.8]
            \addplot[mark=None, color=red] 
                table[x=Step, y=Value, col sep=comma]{images/network_csv/baseline/MobileNetV2/MobileNetV2_baseline_validation_acuracy.csv};
            \addplot[mark=None, color=blue] 
                table[x=Step, y=Value, col sep=comma]{images/network_csv/baseline/baseline_distance/MobileNetV2_baseline_distance_validation_acuracy.csv};
            
            \nextgroupplot[
                grid=major, 
                grid style={dashed,gray!30},
                xlabel=Epoch, % Set the labels
                ylabel=Distance,
                yticklabel pos=right,
                ylabel near ticks]
                \addplot[mark=None, color=red] 
                    table[x=Step, y=Value, col sep=comma]{images/network_csv/baseline/baseline/MobileNetV2_baseline_distance0.csv};
                \addplot[mark=None, color=blue] 
                    table[x=Step, y=Value, col sep=comma]{images/network_csv/baseline/baseline_distance/MobileNetV2_baseline_distance_distance0.csv};
    
            \end{groupplot}
        \end{tikzpicture}
        \caption{Validation accuray (left) and Distance values (right) for a network trained without distance function (red) and with distance function (blue).}
    \end{center}
\end{figure}

\end{comment}



have to do longer run of width and longer run of strength for highest factor at
least, let distance kernel out again and look if comes back, add distance kernel
for 0 epoch to show weight size





road for next week:



mo  go over introduction
tu  and methods
we  Finish results
th  go over results
fr  go over discussion and finish discussion
sa  


look at l2 without und lr0
l2 result seems to confirm view also shows how l2 is important
