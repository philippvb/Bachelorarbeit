\chapter{Results}
After the theoretical introduction in chapter \ref{cha:Methods}, this chapter
provides some empirical results. In section \ref{res:Baseline}, a baseline for
further comparisons will be created. Section \ref{res:Hyperparameters} will
focus on the effects of the hyperparameters of the distance function. How
multiple checkpoints can be used will section \ref{res:Multiple}show, followed
by the combination with learning rate schedulers \ref{res:Learning_rate}. We
will investigate the effects on ensemble methods in section \ref{res:Ensemble}
and finally study the computational cost in \ref{Res:Computational_cost}.



\section{Baseline}\label{res:Baseline} For the baseline, we use the
hyperparameters as defined in section \ref{sub:Hyperparameters}. The distance
function is used with a width of $\sigma^2=1000$ and a strength of $s=1$. First
we train both networks without distance function. After 100 epochs, we add a
checkpoint in the case of the network with distance function.

For the validation accuray, we can see no impact of the distance function. Both
networks show a standard learning curve, with a initial strong increase in
validation accuracy until coming into convergence in later epochs. This results
in a validation accuracy of around 90\% for MobileNetV2 and 89\% for ResNet32
after 600 epochs.

If we plot the distance the network gets to the checkpoint however, we can see
that the network trained with distance function clearly increases its distance
more than the one trained without. Furthermore, the red plot follows the shape
we would expect from the distance function. At the beginning, the distance
increases relatively fast, as can be expected from the high gradient of the RBF
Kernel. But as the distance increases, its gradient becomes smaller, similar to
the gradient of the Kernel. The blue plot follows a smilar curve, but with a
much smaller gradient and overall distance. Therefore, the additional term
succeeds in the initial goal, which was to distance from the checkpoint. Note
that despite the validation accuracy beeing in an area of convergence, SGD even
without distance function keeps on walking and never truly stops at a point.

\begin{figure}[H]\label{fig:Results_baseline}
    \begin{center}
        \begin{tikzpicture}
            \begin{groupplot}[
                group style={
                group size=2 by 2,
                horizontal sep=10pt,
                vertical sep=10pt,
                group name=G},
                width=8cm,
            ]

            \nextgroupplot[
            grid=major, 
            grid style={dashed,gray!30},
            % xlabel=Epoch,
            ylabel=Validation Accuracy,
            xticklabels={,,},
            ymin=0.85,
            xmin=-10]
            \addplot[mark=None, color=red] 
                table[x=Step, y=Value, col sep=comma]{images/network_csv/baseline/MobileNetV2/MobileNetV2_baseline_validation_acuracy.csv};
            \addplot[mark=None, color=blue] 
                table[x=Step, y=Value, col sep=comma]{images/network_csv/baseline/MobileNetV2/MobileNetV2_baseline_distance_validation_acuracy.csv};
            
            \nextgroupplot[
                grid=major, 
                grid style={dashed,gray!30},
                % xlabel=Epoch,
                ylabel=Distance,
                yticklabel pos=right,
                xticklabels={,,},
                ylabel near ticks]
                \addplot[mark=None, color=red] 
                    table[x=Step, y=Value, col sep=comma]{images/network_csv/baseline/MobileNetV2/MobileNetV2_baseline_distance0.csv};
                \addplot[mark=None, color=blue] 
                    table[x=Step, y=Value, col sep=comma]{images/network_csv/baseline/MobileNetV2/MobileNetV2_baseline_distance_distance0.csv};
    

            \nextgroupplot[
            grid=major, 
            grid style={dashed,gray!30},
            xlabel=Epoch, % Set the labels
            ylabel=Validation Accuracy,
            ymin=0.85,
            xmin=-10]
            \addplot[mark=None, color=red] 
                table[x=Step, y=Value, col sep=comma]{images/network_csv/baseline/ResNet32/ResNet32_baseline_validation_acuracy.csv};
            \addplot[mark=None, color=blue] 
                table[x=Step, y=Value, col sep=comma]{images/network_csv/baseline/ResNet32/ResNet32_baseline_distance_validation_acuracy.csv};
            
            \nextgroupplot[
                grid=major, 
                grid style={dashed,gray!30},
                xlabel=Epoch, % Set the labels
                ylabel=Distance,
                yticklabel pos=right,
                ylabel near ticks]
                \addplot[mark=None, color=red] 
                    table[x=Step, y=Value, col sep=comma]{images/network_csv/baseline/ResNet32/ResNet32_baseline_distance0.csv};
                \addplot[mark=None, color=blue] 
                    table[x=Step, y=Value, col sep=comma]{images/network_csv/baseline/ResNet32/ResNet32_baseline_distance_distance0.csv};

            \end{groupplot}
        \end{tikzpicture}
        \caption{Validation accuray and distance to the checkpoint for MobileNetV2 (upper) and ResNet32 (lower) trained without distance function (red) and with distance function (blue).}
    \end{center}
\end{figure}



\section{Distance function Hyperparameters}\label{res:Hyperparameters}
\subsection{strength}\label{res:Strength}
Recall section \ref{eq:Loss_strength}, where we added strength as a
hyperparameter, controlling the influence of the distance function. As the RBF
Kernel is bound between 0 and 1, the strength hyperparameter will increase that
bound between 0 and the strength value. In this section, we use the same
hyperparameters as in section \ref{res:Baseline} but vary the strength
parameter. We also add the checkpoint after 100 epochs.

For higher strength values, the validation accuracy receives an initial drop
after a checkpoint is added, which is larger the higher the strength parameter
value, see figure \ref{fig:Results_strength}. This may be due to the increased
influence of the strength function. Each distance between the parameters and the
checkpoint starts at 0, therefore the values of the exponential function starts
at $strength \cdot 1$. A higher strength value will lead to a higher influence
of the distanc term on the loss function, and consequently on the gradient. The
gradient of the cross-entropy loss will become insignificant, and the optimizer
will update the weights without regards to the validation accuracy, hence the
drop at the beginning. 

The distance plot reflects this behaviour, as we can see an larger increase in
distance for larger strength values. After the initial step however, the
distance quickly converges. This is due to the fact that after the initial
strong decrease, the value of the distance function quickly comes close to 0.
Therefore, the distance Kernel becomes insignificant again, and the
Cross-Entropy loss is followed. The validation accuracy reflects this behaviour,
as after the initial drop the accuracy recovers to the level of before. This
provides additional insight in the loss landscape. As the network distances from
it`s current position on the loss landscpae, but is still able to reach high
accuracy, there have to be areas of high accuracy everywhere on the landscape.
This is in consonance to other research in this field, as discussed in section
\ref{loss_landscape}.

As a result, the value of the strength doesn't matter in long term at least. It
merely increases the distance to the checkpoint by defining how long the
distance term is important to the loss function. But after that, the normal loss
is followed again. As we above suggested, areas of low loss and high accuray can
be found everywhere. Therefore, the same accuracy as before can be reached, no
matter the value of the strength. On the other hand, it is also unlikely that it
will outperform the last best value, as new areas are not generally better than
other. That's why the effect of the distance doesn't transfer to the validation
accuracy.



\begin{figure}[h]\label{fig:Results_strength}
    \begin{center}
        \begin{tikzpicture}
            \begin{groupplot}[
                group style={
                group size=2 by 2,
                horizontal sep=10pt,
                vertical sep=10pt,
                group name=G},
                width=8cm,
                restrict x to domain=0:200
            ]

            \nextgroupplot[
            grid=major, 
            grid style={dashed,gray!30},
            % xlabel=Epoch,
            y label style={at={(axis description cs:0.1,0)},anchor=south},
            ylabel=Validation Accuracy,
            xticklabels={,,},
            ymin=0.7,
            legend pos= south east,
            xmin=-10]
            \addplot[mark=None, color=red] 
                table[x=Step, y=Value, col sep=comma]{images/network_csv/strength/MobileNetV2/MobileNetV2_strength_e1_validation_acuracy.csv};
            \addlegendentry{$s=0.1$}
            \addplot[mark=None, color=blue] 
                table[x=Step, y=Value, col sep=comma]{images/network_csv/baseline/MobileNetV2/MobileNetV2_baseline_distance_validation_acuracy.csv};
            \addlegendentry{$s=1$}
            \addplot[mark=None, color=orange] 
                table[x=Step, y=Value, col sep=comma]{images/network_csv/strength/MobileNetV2/MobileNetV2_strength_e2_validation_acuracy.csv};
            \addlegendentry{$s=10$}
            \addplot[mark=None, color=green] 
                table[x=Step, y=Value, col sep=comma]{images/network_csv/strength/MobileNetV2/MobileNetV2_strength_e3_validation_acuracy.csv};
            \addlegendentry{$s=100$}

            \nextgroupplot[
                grid=major, 
                grid style={dashed,gray!30},
                % xlabel=Epoch,
                y label style={at={(-0.1,0.5)},anchor=south},
                ylabel=Distance,
                yticklabel pos=right,
%                y label style={at={(axis description cs:0,0)},anchor=south},
                xticklabels={,,},
                ylabel near ticks]
            \addplot[mark=None, color=red] 
                table[x=Step, y=Value, col sep=comma]{images/network_csv/strength/MobileNetV2/MobileNetV2_strength_e1_distance0.csv};
            \addplot[mark=None, color=blue] 
                table[x=Step, y=Value, col sep=comma]{images/network_csv/baseline/MobileNetV2/MobileNetV2_baseline_distance_distance0.csv};
            \addplot[mark=None, color=orange] 
                table[x=Step, y=Value, col sep=comma]{images/network_csv/strength/MobileNetV2/MobileNetV2_strength_e2_distance0.csv};
            \addplot[mark=None, color=green] 
                table[x=Step, y=Value, col sep=comma]{images/network_csv/strength/MobileNetV2/MobileNetV2_strength_e3_distance0.csv};
    

            \nextgroupplot[
            grid=major, 
            grid style={dashed,gray!30},
            % x label style={at={(axis description cs:1.5,0)},anchor=west},
            xlabel=Epoch, % Set the labels
            % ylabel=Validation Accuracy,
            ymin=0.7,
            xmin=-10]
            \addplot[mark=None, color=red] 
                table[x=Step, y=Value, col sep=comma]{images/network_csv/strength/ResNet32/ResNet32_strength_e1_validation_acuracy.csv};
            \addplot[mark=None, color=blue] 
                table[x=Step, y=Value, col sep=comma]{images/network_csv/baseline/ResNet32/ResNet32_baseline_distance_validation_acuracy.csv};
            \addplot[mark=None, color=orange] 
                table[x=Step, y=Value, col sep=comma]{images/network_csv/strength/ResNet32/ResNet32_strength_e2_validation_acuracy.csv};
            \addplot[mark=None, color=green] 
                table[x=Step, y=Value, col sep=comma]{images/network_csv/strength/ResNet32/ResNet32_strength_e3_validation_acuracy.csv};
            
            \nextgroupplot[
                grid=major, 
                grid style={dashed,gray!30},
                %xlabel=Epoch, % Set the labels
                ylabel=Distance,
                yticklabel pos=right,
                ylabel near ticks]
            \addplot[mark=None, color=red] 
                table[x=Step, y=Value, col sep=comma]{images/network_csv/strength/ResNet32/ResNet32_strength_e1_distance0.csv};
            \addplot[mark=None, color=blue] 
                table[x=Step, y=Value, col sep=comma]{images/network_csv/baseline/ResNet32/ResNet32_baseline_distance_distance0.csv};
            \addplot[mark=None, color=orange] 
                table[x=Step, y=Value, col sep=comma]{images/network_csv/strength/ResNet32/ResNet32_strength_e2_distance0.csv};
            \addplot[mark=None, color=green] 
                table[x=Step, y=Value, col sep=comma]{images/network_csv/strength/ResNet32/ResNet32_strength_e3_distance0.csv};

            \end{groupplot}
        \end{tikzpicture}
        \caption{Validation accuracy and Distance for MobileNetV2 (upper) and ResNet32 (lower) trained without distance function and different strength values.}
    \end{center}
\end{figure}

\subsection{width}
We have seen in chapter \ref{distance_function}, how the width $\sigma$
influences the distance function: The larger $\sigma$ gets, the wider it
becomes. Therefore, a larger $\sigma$ should result in the network distancing
further from it's checkpoint than for smaller values. That's exactly what can be
seen in figure \ref{fig:Results_width}: For an $\sigma^2$ of 0.1, the distance
follows the one of network trained without distance function, due to the
distance function quickly becoming 0 and therefore having no influence. The
larger the width, the slower the values of the distance function decrease for further
distance. Therefore, the gradient of the distance function will stay quite
large, pushing the network further away. This can be seen for higher $\sigma$
values, where the distance increases further.


Initially however, the distance plot doesn't reflect this behaviour. The curve
for $\sigma^2 = 10$ has a higher distance value than for $\sigma^2 = 100$.
That's because the gradient of the distance function is higher for small distances, the
smaller $\sigma^2$. But for larger distances, this effect flips with the gradient
of the larger width staying higher for more epochs. Therefore, the curves
intersect after additional epochs. The smaller width converges, while the larger
stays constant until eventually reaching it`s area of convergence. The
intersection happens later the larger the width, for $\sigma = 10000$, only
after 307 epochs.

This behaviour however has no influence on the validation accuracy, which stays
the same for every width, in contrast to the strength. This may be due to the
fact, that a larger width does in fact decreases the size of the gradient. A
smaller gradient over more epochs will be added to the gradient of the
Cross-Entropy Loss. The optimizer therefore nearly keeps on following the normal
gradient and descending the valley of low loss, with a small but steady push to
increase the distance to the checkpoint. Therefore, even if the network is
allowed to move quite freely on the loss surface, it finds areas of low loss and
high accuracy in distant areas of the checkpoint, meaning there has to be good
local minima everywhere on the landscape.

In particular, $\sigma^2 = 1000$ seems to increase the distance far enough in a
small number of epochs. That's why this configuration is used as standard in the
following work.
\pagebreak

\begin{figure}[h]\label{fig:Results_width}
    \begin{center}
        \begin{tikzpicture}
            \begin{groupplot}[
                group style={
                group size=2 by 2,
                horizontal sep=10pt,
                vertical sep=10pt,
                group name=G},
                width=8cm,
                restrict x to domain=0:400
            ]

            \nextgroupplot[
            grid=major, 
            grid style={dashed,gray!30},
            % xlabel=Epoch,
            ylabel=Validation Accuracy,
            xticklabels={,,},
            ymin=0.8,
            legend pos = south east,
            xmin=-10]
            \addplot[mark=None, color=red] 
                table[x=Step, y=Value, col sep=comma]{images/network_csv/width/MobileNetV2/MobileNetV2_width_e1_validation_acuracy.csv};
            \addlegendentry{$\sigma^2=10$}
            \addplot[mark=None, color=orange] 
                table[x=Step, y=Value, col sep=comma]{images/network_csv/width/MobileNetV2/MobileNetV2_width_e2_validation_acuracy.csv};
            \addlegendentry{$\sigma^2=100$}
            \addplot[mark=None, color=blue]
                table[x=Step, y=Value, col sep=comma]{images/network_csv/baseline/MobileNetV2/MobileNetV2_baseline_distance_validation_acuracy.csv};
            \addlegendentry{$\sigma^2=1000$}
            \addplot[mark=None, color=green] 
                table[x=Step, y=Value, col sep=comma]{images/network_csv/width/MobileNetV2/MobileNetV2_width_e4_validation_acuracy.csv};
            \addlegendentry{$\sigma^2=10000$}

            \nextgroupplot[
                grid=major, 
                grid style={dashed,gray!30},
                % xlabel=Epoch,
                ylabel=Distance,
                yticklabel pos=right,
                xticklabels={,,},
                ylabel near ticks]
            \addplot[mark=None, color=red] 
                table[x=Step, y=Value, col sep=comma]{images/network_csv/width/MobileNetV2/MobileNetV2_width_e1_distance0.csv};
            \addplot[mark=None, color=orange] 
                table[x=Step, y=Value, col sep=comma]{images/network_csv/width/MobileNetV2/MobileNetV2_width_e2_distance0.csv};
            \addplot[mark=None, color=blue] 
                table[x=Step, y=Value, col sep=comma]{images/network_csv/baseline/MobileNetV2/MobileNetV2_baseline_distance_distance0.csv};
            \addplot[mark=None, color=green] 
                table[x=Step, y=Value, col sep=comma]{images/network_csv/width/MobileNetV2/MobileNetV2_width_e4_distance0.csv};
    

            \nextgroupplot[
            grid=major, 
            grid style={dashed,gray!30},
            xlabel=Epoch, % Set the labels
            ylabel=Validation Accuracy,
            ymin=0.8,
            xmin=-10]
            \addplot[mark=None, color=red] 
                table[x=Step, y=Value, col sep=comma]{images/network_csv/width/ResNet32/ResNet32_width_e1_validation_acuracy.csv};
            \addplot[mark=None, color=orange] 
                table[x=Step, y=Value, col sep=comma]{images/network_csv/width/ResNet32/ResNet32_width_e2_validation_acuracy.csv};
            \addplot[mark=None, color=blue] 
                table[x=Step, y=Value, col sep=comma]{images/network_csv/baseline/ResNet32/ResNet32_baseline_distance_validation_acuracy.csv};
            \addplot[mark=None, color=green] 
                table[x=Step, y=Value, col sep=comma]{images/network_csv/width/ResNet32/ResNet32_width_e4_validation_acuracy.csv};

            
            \nextgroupplot[
                grid=major, 
                grid style={dashed,gray!30},
                xlabel=Epoch, % Set the labels
                ylabel=Distance,
                yticklabel pos=right,
                ylabel near ticks]
            \addplot[mark=None, color=red] 
                table[x=Step, y=Value, col sep=comma]{images/network_csv/width/ResNet32/ResNet32_width_e1_distance0.csv};
            \addplot[mark=None, color=orange] 
                table[x=Step, y=Value, col sep=comma]{images/network_csv/width/ResNet32/ResNet32_width_e2_distance0.csv};
            \addplot[mark=None, color=blue] 
                table[x=Step, y=Value, col sep=comma]{images/network_csv/baseline/ResNet32/ResNet32_baseline_distance_distance0.csv};
            \addplot[mark=None, color=green] 
                table[x=Step, y=Value, col sep=comma]{images/network_csv/width/ResNet32/ResNet32_width_e4_distance0.csv};

            \end{groupplot}
        \end{tikzpicture}
        \caption{Validation accuray and Distance for MobileNetV2 (upper) and ResNet32 (lower) trained with different widths of the distance function.}
    \end{center}
\end{figure}



\section{Multiple Checkpoints}\label{res:Multiple} 
In chapter \ref{cha:Methods}, we have seen that we can not only incorporate one
but also multiple checkpoints in the loss function, on which this section will
focus. Here, we add a checkpoint after every 150 epochs that are performed,
resulting in a total of 3 checkpoints after 600 epochs. We also perform the same
analysis as in section \ref{res:Strength}, where we alternate the strength
factor.

For the validation accuracy, we can see the same pattern as in section
\ref{res:Hyperparameters}. For low strength, we experience no effect at all. If
we increase the strength there is an initial drop in the accuracy. For
MobileNetV2, the accuracy recovers and even tops the baseline, with a small but
increasing margin for every new checkpoints that is added, coming to an
improvement of 0.8\% over the baseline after 600 epochs. In contrast, ResNet32
doesn't seem to benefit from multiple checkpoints, the accuracy even decreases
for larger strength values.

The pattern of MobileNetV2 is quite similar to warm restarts, where for every
restart the reached accuracy increases. For a in depth comparison, see section
\ref{res:Scheduler}.

The influence on the distance to the checkpoints is more interesting however. In
general, an additional checkpoint seems to decrease the distance to the others.
This is quite counterintuitive at first glance. Suppose for one parameter, we
have checkpoint value $a$ and create a new one at our current value $b<a$. To
increase the distance to both, we would just have to walk in direction $c<b$. As
we have seen in previous section, this would be sufficient as areas of low error
exist everywhere. But why is the distance then decreased for our first
checkpoint? The reason might be the $L_2$ regularization, as it restricts the
weights to small sizes. At some point, it might be less costly to decrease the
weights again for a smaller $L_2$, traded off against a bit smaller distance to
other checkpoints. This restriction in weight space is probably why new terms
lead to a decrease of the distance to existing checkpoints. A network trained
without $L_2$ regularization futher supports this explanation, see figure \ref{fig:Noreg}. Here, the
distance increases further for each new checkpoint that is added.


\begin{figure}[h]\label{fig:Noreg}
    \begin{center}
        \begin{tikzpicture}
            \begin{axis}[
                grid=major, 
                grid style={dashed,gray!30},
                xlabel=Epoch,
                ylabel=Distance,
                ylabel near ticks,
                xmin=140,
                width =8cm]
                \addplot[mark=None, color=blue] 
                table[x=Step, y=Value, col sep=comma]{images/network_csv/multiple/MobileNetV2/MobileNetV2_multiple_f100_noreg_distance0.csv};
            \end{axis}
        \end{tikzpicture}
        \caption{Distance to the first checkpoint after 150 epochs for MobileNetV2 trained with multiple checkpoints and $s=100$, but without $L_2$ regularization.}
    \end{center}
\end{figure}

\begin{figure}[h]\label{fig:Results_multiple}
    \begin{center}
        \begin{tikzpicture}
            \begin{groupplot}[
                group style={
                group size=2 by 1,
                horizontal sep=10pt,
                vertical sep=10pt,
                group name=G},
                width=8cm
            ]

            \nextgroupplot[
            grid=major, 
            grid style={dashed,gray!30},
            xlabel=Epoch,
            ylabel=Validation Accuracy,
            ymin=0.7,
            ymax=0.95,
            legend pos = south east,
            xmin=-10]
            \addplot[mark=None, color=red] 
                table[x=Step, y=Value, col sep=comma]{images/network_csv/multiple/MobileNetV2/MobileNetV2_multiple_validation_acuracy.csv};
                \addlegendentry{$s=1$}
            \addplot[mark=None, color=green] 
                table[x=Step, y=Value, col sep=comma]{images/network_csv/multiple/MobileNetV2/MobileNetV2_multiple_f10_validation_acuracy.csv};
                \addlegendentry{$s=10$}
            \addplot[mark=None, color=blue] 
                table[x=Step, y=Value, col sep=comma]{images/network_csv/multiple/MobileNetV2/MobileNetV2_multiple_f100_validation_acuracy.csv};
                \addlegendentry{$s=100$}
            
    

            \nextgroupplot[
            grid=major, 
            grid style={dashed,gray!30},
            xlabel=Epoch, % Set the labels
            ytick pos=right,
            ymin=0.7,
            ymax=0.95,
            xmin=-10]
            \addplot[mark=None, color=red] 
                table[x=Step, y=Value, col sep=comma]{images/network_csv/multiple/ResNet32/ResNet32_multiple_validation_acuracy.csv};
            \addplot[mark=None, color=green] 
                table[x=Step, y=Value, col sep=comma]{images/network_csv/multiple/ResNet32/ResNet32_multiple_f10_validation_acuracy.csv};
            \addplot[mark=None, color=blue] 
                table[x=Step, y=Value, col sep=comma]{images/network_csv/multiple/ResNet32/ResNet32_multiple_f100_validation_acuracy.csv};
            % [TODO: add ResNet Multiple]

            \end{groupplot}
        \end{tikzpicture}
        \caption{Validation accuray and Distance for MobileNetV2 (left) and ResNet32 (right) trained with multiple checkpoints.}
    \end{center}
\end{figure}

\begin{figure}[H]\label{fig:Results_multiple_distance}
    \begin{center}
        \begin{tikzpicture}
            \begin{groupplot}[
                group style={
                group size=3 by 2,
                horizontal sep=10pt,
                vertical sep=10pt,
                group name=G},
                width=5cm
            ]

            \nextgroupplot[
            title=checkpoint 1,
            grid=major, 
            grid style={dashed,gray!30},
            %x label style={at={(axis description cs:1.5,0)},anchor=north},
            % y label style={at={(axis description cs:-0.1,.5)},rotate=90,anchor=south},
            %xlabel=Epoch,
            ylabel=Distance,
            ylabel near ticks,
            ymax=8000,
            xmin=140,
            xticklabels={,,}]
            \addplot[mark=None, color=red] 
                table[x=Step, y=Value, col sep=comma]{images/network_csv/multiple/MobileNetV2/MobileNetV2_multiple_distance0.csv};
            \addplot[mark=None, color=green] 
                table[x=Step, y=Value, col sep=comma]{images/network_csv/multiple/MobileNetV2/MobileNetV2_multiple_f10_distance0.csv};
            \addplot[mark=None, color=blue] 
                table[x=Step, y=Value, col sep=comma]{images/network_csv/multiple/MobileNetV2/MobileNetV2_multiple_f100_distance0.csv};

            \nextgroupplot[
            title=checkpoint 2,
            grid=major, 
            grid style={dashed,gray!30},
            yticklabels={,,}
            ymax=8000,
            xmin=290,
            xticklabels={,,}]
            \addplot[mark=None, color=red] 
                table[x=Step, y=Value, col sep=comma]{images/network_csv/multiple/MobileNetV2/MobileNetV2_multiple_distance1.csv};
            \addplot[mark=None, color=green] 
                table[x=Step, y=Value, col sep=comma]{images/network_csv/multiple/MobileNetV2/MobileNetV2_multiple_f10_distance1.csv};
            \addplot[mark=None, color=blue] 
                table[x=Step, y=Value, col sep=comma]{images/network_csv/multiple/MobileNetV2/MobileNetV2_multiple_f100_distance1.csv};

            \nextgroupplot[
            title=checkpoint 3,
            grid=major, 
            grid style={dashed,gray!30},
            yticklabels={,,},
            ymax=8000,
            legend pos = outer north east,
            xmin=440,
            xticklabels={,,}]
            \addplot[mark=None, color=red] 
                table[x=Step, y=Value, col sep=comma]{images/network_csv/multiple/MobileNetV2/MobileNetV2_multiple_distance2.csv};
                \addlegendentry{$s=1$}
            \addplot[mark=None, color=green] 
                table[x=Step, y=Value, col sep=comma]{images/network_csv/multiple/MobileNetV2/MobileNetV2_multiple_f10_distance2.csv};
                \addlegendentry{$s=10$}
            \addplot[mark=None, color=blue] 
                table[x=Step, y=Value, col sep=comma]{images/network_csv/multiple/MobileNetV2/MobileNetV2_multiple_f100_distance2.csv};
                \addlegendentry{$s=100$}

%--------------------ResNet----------------------------------------------------

            \nextgroupplot[
            grid=major, 
            grid style={dashed,gray!30},
            ylabel=Distance,
            ylabel near ticks,
            ymax=8000,
            xmin=140]
            \addplot[mark=None, color=red] 
                table[x=Step, y=Value, col sep=comma]{images/network_csv/multiple/ResNet32/ResNet32_multiple_distance0.csv};
            \addplot[mark=None, color=green] 
                table[x=Step, y=Value, col sep=comma]{images/network_csv/multiple/ResNet32/ResNet32_multiple_f10_distance0.csv};
            \addplot[mark=None, color=blue] 
                table[x=Step, y=Value, col sep=comma]{images/network_csv/multiple/ResNet32/ResNet32_multiple_f100_distance0.csv};

            \nextgroupplot[
            grid=major, 
            grid style={dashed,gray!30},
            xlabel=Epoch,
            yticklabels={,,}
            ymax=8000,
            xmin=290]
            \addplot[mark=None, color=red] 
                table[x=Step, y=Value, col sep=comma]{images/network_csv/multiple/ResNet32/ResNet32_multiple_distance1.csv};
            \addplot[mark=None, color=green] 
                table[x=Step, y=Value, col sep=comma]{images/network_csv/multiple/ResNet32/ResNet32_multiple_f10_distance1.csv};
            \addplot[mark=None, color=blue] 
                table[x=Step, y=Value, col sep=comma]{images/network_csv/multiple/ResNet32/ResNet32_multiple_f100_distance1.csv};

            \nextgroupplot[
            grid=major, 
            grid style={dashed,gray!30},
            yticklabels={,,},
            ymax=8000,
            xmin=440]
            \addplot[mark=None, color=red] 
                table[x=Step, y=Value, col sep=comma]{images/network_csv/multiple/ResNet32/ResNet32_multiple_distance2.csv};
            \addplot[mark=None, color=green] 
                table[x=Step, y=Value, col sep=comma]{images/network_csv/multiple/ResNet32/ResNet32_multiple_f10_distance2.csv};
            \addplot[mark=None, color=blue] 
                table[x=Step, y=Value, col sep=comma]{images/network_csv/multiple/ResNet32/ResNet32_multiple_f100_distance2.csv};

            \end{groupplot}
        \end{tikzpicture}
        \caption{Distance plots for MobileNetV2 (upper) and ResNet32 (lower) trained with multiple checkpoints and different strength values.}
    \end{center}
\end{figure}

In short term however, new checkpoint produce quite different influence on the
distance to existing ones. For a strength of $s=1$ we can see a slight decrease
which turns into a longer increase until the next checkpoint. For $s=10$, we see
an aprupt decrease with an small peak after. For $s=100$, there only is a peak
at the beginning which quickly diminishes. Noticeable, this pattern repeats for
the other checkpoints and therefore seems unlikely to be a coincidence. The size
of these hills can be tracked to the size of the strength. Larger strength
introduces a larger gradient and therefore step size at the beginning, making
the hills larger. This doesn't explain the different orientation of these hills.
For the case of $s=100$, the initial increase of distance seems reasonable. As
the value of the old distance function is not 0 when a new checkpoints is
created, the current gradient should face in direction away from the last
checkpoint. A new checkpoint term in the loss function introduces a gradient
boost, as section \ref{sub:Effect_on_Gradient} discussed. Because momentum
preserves the old gradient, it will acclerate in direction away from the first
and the second checkpoint, therefore leading to a hill. However by this
explanation, the pattern should repeat for the other strength values, which is
not the case. Therefore, the true cause of the difference in short term effects
remains unclear.



\section{Learning rate}\label{res:Learning_rate}
\subsection{scheduler}\label{res:Scheduler}
Learning rate Scheduler were introduced in section \ref{sub:Learing_rate_decay},
where the learning rate was not held constant, but decayed over time. One
further addition were warm restarts, where after the decay we set the learning
rate back up to the inital and repeat this cycle several times. We use a step
decay scheduler with $\gamma = 0.1$ for every 50 epochs, and a warm restart
after 150 epochs. For cosine decay with warm restart, we set $\epsilon_{min}=0$,
$\epsilon_{max}=0.01$ and $T_0=150$.


For the validation accuracy, schedulers outperform a fixed learning rate as we
can see in figure \ref{fig:Results_scheduler}. At the beginning of every cycle,
the performance drops due to high learning rate. But when the learning rate
decays again, the accuracy recovers. Furthermore, the maximum accuracy of every
cycle slightly increases, until eventually coming to convergence. Here, a cosine
decay outperforms a step decay. That's probably due to the smaller learning rate
at the end fo each cycle cosine decay can reach. This leads to a maximum
accuracy of around 95\% for MobileNetV2 and 93\% for ResNet32.


\begin{figure}[h]\label{fig:Results_scheduler}
    \begin{center}
        \begin{tikzpicture}
            \begin{groupplot}[
                group style={
                group size=2 by 2,
                horizontal sep=10pt,
                vertical sep=10pt,
                group name=G},
                width=8cm
            ]

            \nextgroupplot[
            grid=major, 
            grid style={dashed,gray!30},
            % xlabel=Epoch,
            ylabel=Validation Accuracy,
            xticklabels={,,},
            ymin=0.85,
            xmin=-10]
            \addplot[mark=None, color=red] 
                table[x=Step, y=Value, col sep=comma]{images/network_csv/baseline/MobileNetV2/MobileNetV2_baseline_validation_acuracy.csv};
            \addplot[mark=None, color=green] 
                table[x=Step, y=Value, col sep=comma]{images/network_csv/scheduler/MobileNetV2/MobileNetV2_scheduler_step_validation_acuracy.csv};
            \addplot[mark=None, color=blue] 
                table[x=Step, y=Value, col sep=comma]{images/network_csv/scheduler/MobileNetV2/MobileNetV2_scheduler_cosine_validation_acuracy.csv};



            \nextgroupplot[
                grid=major, 
                grid style={dashed,gray!30},
                % xlabel=Epoch,
                ylabel=Distance,
                yticklabel pos=right,
                xticklabels={,,},
                legend pos = north west,
                legend style={nodes={scale=0.8, transform shape}},
                ylabel near ticks]
            \addplot[mark=None, color=red, x filter/.code=\pgfmathparse{\pgfmathresult+50}]
                table[x=Step, y=Value, col sep=comma]{images/network_csv/baseline/MobileNetV2/MobileNetV2_baseline_distance0.csv};
                \addlegendentry{fixed lr}
            \addplot[mark=None, color=green] 
                table[x=Step, y=Value, col sep=comma]{images/network_csv/scheduler/MobileNetV2/MobileNetV2_scheduler_step_distance0.csv};
                \addlegendentry{step decay}
            \addplot[mark=None, color=blue] 
                table[x=Step, y=Value, col sep=comma]{images/network_csv/scheduler/MobileNetV2/MobileNetV2_scheduler_cosine_distance0.csv};
                \addlegendentry{cosine decay}
    

            \nextgroupplot[
            grid=major, 
            grid style={dashed,gray!30},
            xlabel=Epoch, % Set the labels
            ylabel=Validation Accuracy,
            ymin=0.85,
            xmin=-10]
            \addplot[mark=None, color=red]
                table[x=Step, y=Value, col sep=comma]{images/network_csv/baseline/ResNet32/ResNet32_baseline_validation_acuracy.csv};
            \addplot[mark=None, color=green] 
               table[x=Step, y=Value, col sep=comma]{images/network_csv/scheduler/ResNet32/ResNet32_scheduler_step_validation_acuracy.csv};
            \addplot[mark=None, color=blue] 
                table[x=Step, y=Value, col sep=comma]{images/network_csv/scheduler/ResNet32/ResNet32_scheduler_cosine_validation_acuracy.csv};


            
            \nextgroupplot[
                grid=major, 
                grid style={dashed,gray!30},
                xlabel=Epoch, % Set the labels
                ylabel=Distance,
                yticklabel pos=right,
                ylabel near ticks]
            \addplot[mark=None, color=red, x filter/.code=\pgfmathparse{\pgfmathresult+50}] 
                table[x=Step, y=Value, col sep=comma]{images/network_csv/baseline/ResNet32/ResNet32_baseline_distance0.csv};
            \addplot[mark=None, color=green] 
                table[x=Step, y=Value, col sep=comma]{images/network_csv/scheduler/ResNet32/ResNet32_scheduler_step_distance0.csv};
            \addplot[mark=None, color=blue] 
                table[x=Step, y=Value, col sep=comma]{images/network_csv/scheduler/ResNet32/ResNet32_scheduler_cosine_distance0.csv};
            \end{groupplot}
        \end{tikzpicture}
        \caption{Validation accuray and Distance to a checkpoint after 100 epochs for MobileNetV2 (upper) and ResNet32 (lower) trained with learning rate schedulers but without distance function.}
    \end{center}
\end{figure}


The distance plot now also looks a bit different compared to a fixed lr. We can
see that after an initial increase in distance to the checkpoint, the distance
reaches a local maximum and then actually decreases again until coming into an
area of convergence. For each restart performed, the same pattern repeats again.
For step decay, we can see hard edges in contrast to smooth curves of cosine
decay. The similarity to the alteration of the learning rate gives evidence that
the decrease of distance is caused by the learning rate. 

One possible explanation is that the decrease in learnig rate could lead to
smaller increase in distance by the pure fact, that smaller learning rate leads
to smaller updates of the weights and threfeore smaller increase in distance.
However, this couldn`t explain why the distance itself actually keeps
decreasing, not just smaller increasing. Only could try to explain this with the
$L_2$ Loss and a similar argument to section \ref{res:Multiple}. For high
learning rate, a larger Momentum builds up. This leads to parameter values,
which are very larger. After the momentum is slowed down, the $L_2$
regularization keeps decreasing the weights again until reaching an equilibrium.
But larger convergences in the following cycles make this explanation doubtful. 


We have seen that whenever a warm restart is performed, the accuracy drops.
However with decreasing learning rate, the network recovers and even tops the
performance of the previous cycle. This pattern is quite similar to effect of
multiple checkpoint of section \ref{res:Multiple}. Section
\ref{sub:Effect_on_Gradient} further motivates a comparison: We suggested that a
checkpoint would initially just increase the size of the gradient, rather than
changing its orientation, given that the orientation is stable over some epochs.
Therefore, the same effect on the parameter update, but rather from a larger
gradient itself than a larger learning rate, should be achieved.

Figure \ref{fig:Cosine_Multiple} shows a comparison of a cosine decay with warm
restart for a network trained with distance function, multiple checkpoints and
$s=100$ from section \ref{res:Multiple}. As mentioned, the pattern is quite
similar, with a drop in validation accuracy after the warm restart or
checkpoint. The top accuracy in each cycle also tops the last one for both
cases. The effect for MobileNetV2 is with around 1\% against 0.8\% increase in
maximum accuracy between the first and last cycle for cosine decay stronger than
for multiple checkpoints. Additionallly, the network with cosine decay gets a
much better absolute accuracy. That's probably due to the small learning rate at
the end of each cycle, where the network can exploit a small region to fine
adjust the parameter values.

In contrast to MobileNetV2, section \ref{res:Multiple} showed that ResNet32
didn't benefit from multiple checkpoints. If the larger update step is responsible
for the boost in performance for each cycle, then ResNet32 should benefit the
same as MobileNetV2. Because this is not the case, it remains unclear if either
the explanation is wrong or if some properties of the ResNet32 architecture
prevent the effect.

\begin{figure}[h]\label{fig:Cosine_Multiple}
    \begin{center}
        \begin{tikzpicture}
            \begin{groupplot}[
                group style={
                group size=2 by 1,
                horizontal sep=10pt,
                vertical sep=10pt,
                group name=G},
                width=8cm
            ]
                \nextgroupplot[
                    grid=major, 
                    grid style={dashed,gray!30},
                    xlabel=Epoch,
                    ylabel=Validation Accuracy,
                    ymin=0.75,
                    ymax=0.97,
                    legend pos = south east,%#outer north east,
                    xmin=-10]
                    
                    \addplot[mark=None, color=red] 
                        table[x=Step, y=Value, col sep=comma]{images/network_csv/scheduler/MobileNetV2/MobileNetV2_scheduler_cosine_validation_acuracy.csv};
                        \addlegendentry{cosine decay}
                    \addplot[mark=None, color=green] 
                        table[x=Step, y=Value, col sep=comma]{images/network_csv/multiple/MobileNetV2/MobileNetV2_multiple_f100_validation_acuracy.csv};
                        \addlegendentry{multiple checkpoints}

                \nextgroupplot[
                    grid=major, 
                    grid style={dashed,gray!30},
                    xlabel=Epoch,
                    ymin=0.75,
                    ymax=0.97,
                    yticklabels={,,},
                    ylabel = Validation Accuracy,
                    yticklabel pos=right,
                    ylabel near ticks,
                    xmin=-10]
                    
                    \addplot[mark=None, color=red] 
                        table[x=Step, y=Value, col sep=comma]{images/network_csv/scheduler/ResNet32/ResNet32_scheduler_cosine_validation_acuracy.csv};
                    \addplot[mark=None, color=green] 
                        table[x=Step, y=Value, col sep=comma]{images/network_csv/multiple/ResNet32/ResNet32_multiple_f100_validation_acuracy.csv};
            \end{groupplot}        
        \end{tikzpicture}
        \caption{Validation accuracy of multiple checkpoints compared to cosine decay with warm restart for MobileNetV2(left) and ResNet32(right).}
    \end{center}
\end{figure}

Would a combination of both techniques lead to an even better result? Here, we
combined the cosine decay with multiple checkpoints at each warm restart. We can
see that this decreases the top validation accuracy of each cycle. Furthermore,
the distance behaves differently when compared to cosine decay, at least in the
first cycle. Where normal cosine decay decreases the distance again for small
learning rates, the distance function prevents this from happening. Instead, it
just normally converges. In later cycles however, the distance kernel becomes
insignificant again and the patterns tend to look more similar.

If we compare it to multiple checkpoints of section \ref{res:Multiple}, the
distance plot resembles more the shape of the cosine decay. Where for a fixed
learning rate and multiple checkpoints, each new checkpoint led to an initial
decrease and then increase in distance, cosine decay and distance function
actually turn this effect around. Here, an inital increase is followed by a
decrease. As stated above, this also happens without distance function.
Therefore it seems, that the learning rate has a potentially stronger influence
on the shape of the distance to one checkpoint than other checkpoints of the
distance function itself.

\begin{figure}[h]\label{fig:Cosine_distance}
        \begin{center}
            \begin{tikzpicture}
                \begin{groupplot}[
                    group style={
                    group size=2 by 2,
                    horizontal sep=10pt,
                    vertical sep=10pt,
                    group name=G},
                    width=8cm
                ]
    
                \nextgroupplot[
                grid=major, 
                grid style={dashed,gray!30},
                % xlabel=Epoch,
                ylabel=Validation Accuracy,
                xticklabels={,,},
                ymin=0.75,
                legend pos = south east,
                xmin=-10]
                \addplot[mark=None, color=red] 
                    table[x=Step, y=Value, col sep=comma]{images/network_csv/scheduler/MobileNetV2/MobileNetV2_scheduler_cosine_validation_acuracy.csv};
                    \addlegendentry{cosine decay}
                \addplot[mark=None, color=green] 
                    table[x=Step, y=Value, col sep=comma]{images/network_csv/scheduler/MobileNetV2/MobileNetV2_scheduler_cosine_distance_validation_acuracy.csv};
                    \addlegendentry{cosine decay with distance}
                \addplot[mark=None, color=blue] 
                    table[x=Step, y=Value, col sep=comma]{images/network_csv/multiple/ResNet32/ResNet32_multiple_f100_validation_acuracy.csv};
                    \addlegendentry{multiple checkpoints, $s=100$}

                \nextgroupplot[
                    grid=major, 
                    grid style={dashed,gray!30},
                    % xlabel=Epoch,
                    ylabel=Distance,
                    yticklabel pos=right,
                    xticklabels={,,},
                    ylabel near ticks]
                \addplot[mark=None, color=red] 
                    table[x=Step, y=Value, col sep=comma]{images/network_csv/scheduler/MobileNetV2/MobileNetV2_scheduler_cosine_distance0.csv};
                \addplot[mark=None, color=green] 
                    table[x=Step, y=Value, col sep=comma]{images/network_csv/scheduler/MobileNetV2/MobileNetV2_scheduler_cosine_distance_distance0.csv};
                \addplot[mark=None, color=blue] 
                    table[x=Step, y=Value, col sep=comma]{images/network_csv/multiple/MobileNetV2/MobileNetV2_multiple_distance0.csv};
        
    
                \nextgroupplot[
                grid=major, 
                grid style={dashed,gray!30},
                xlabel=Epoch, % Set the labels
                ylabel=Validation Accuracy,
                ymin=0.75,
                xmin=-10]
                \addplot[mark=None, color=red] 
                    table[x=Step, y=Value, col sep=comma]{images/network_csv/scheduler/ResNet32/ResNet32_scheduler_cosine_validation_acuracy.csv};
                \addplot[mark=None, color=green] 
                    table[x=Step, y=Value, col sep=comma]{images/network_csv/scheduler/ResNet32/ResNet32_scheduler_cosine_distance_validation_acuracy.csv};
                \addplot[mark=None, color=blue] 
                    table[x=Step, y=Value, col sep=comma]{images/network_csv/multiple/ResNet32/ResNet32_multiple_f100_validation_acuracy.csv};
    
    
                
                \nextgroupplot[
                    grid=major, 
                    grid style={dashed,gray!30},
                    xlabel=Epoch, % Set the labels
                    ylabel=Distance,
                    yticklabel pos=right,
                    ylabel near ticks]
                \addplot[mark=None, color=red] 
                    table[x=Step, y=Value, col sep=comma]{images/network_csv/scheduler/ResNet32/ResNet32_scheduler_cosine_distance0.csv};
                \addplot[mark=None, color=green] 
                    table[x=Step, y=Value, col sep=comma]{images/network_csv/scheduler/ResNet32/ResNet32_scheduler_cosine_distance_distance0.csv};
                \addplot[mark=None, color=blue] 
                    table[x=Step, y=Value, col sep=comma]{images/network_csv/multiple/ResNet32/ResNet32_multiple_distance0.csv};
                \end{groupplot}
            \end{tikzpicture}
            \caption{Validation accuray and Distance to a checkpoint after 150 epochs for MobileNetV2 (upper) and ResNet32 (lower) trained with learning rate schedulers and without (red) and with (green) distance function with multiple checkpoints.}
        \end{center}
    \end{figure}







\subsection{wrong lr}

We might suspect that a learning rate decay should be more robust to a
suboptimal inital learning rate. At first, this seems to be case. Where
MobileNetV2 without decay only reaches a accuracy of 60\% and then decreases, a
scheduler reaches a comparable accuracy to the best network. But if we perform a
warm restart, the maximum accuracy drops for each cycle, so the network gets
worse with increasing cycles. If we add the distance function however, we can
see that it stabilizes the training so that the maximum accuracy stays the same
for each cycle. This leads to a performance difference of 8\% for MobileNetV2 in
favour of the distance function after 600 epochs. The difference is also present
from the beginning of each restart and remains stable. However, we loose the
effect from above, that each warm restart boosts the performance. Nevertheless,
this is a significant improvement, which is stable across both ResNet32 and
MobileNetV2.

If we compare the distance to the network with optimal learning rate in figure
\ref{fig:Cosine_distance}, there also are some differences. Here, especially
MobileNetV2 fails do distance from its checkpoint at all in contrast to the
optimal learning rate. With the distance function, the distance again goes on to
reduce significantly after the initia increase instead of further increasing.
This leads to the conclusion, that the difference in performance might arise
from the fact, that the network is stuck in an area of high loss. Therefore, the
distance function helps distancing from this area into an area of better
performance. However, this interpretation would contrast other findings from
both our results and the paper, namely that areas of low loss exist everywhere.
In addition, the question why this effect happens for a larger inital learning
rate is still open.

For other learning rates, there is no benefit from the distance function. For a
smaller learning rate, the network performs the same same as normal, even though
a slower convergence at the beginning. For a learning rate of 1, both networks
doesn't learn at all. Instead, the validation accuracy stays constant at around
10\%.

\begin{figure}[h]\label{fig:Results_wrong_lr}
    \begin{center}
        \begin{tikzpicture}
            \begin{groupplot}[
                group style={
                group size=2 by 2,
                horizontal sep=10pt,
                vertical sep=10pt,
                group name=G},
                width=8cm
            ]

            \nextgroupplot[
            grid=major, 
            grid style={dashed,gray!30},
            % xlabel=Epoch,
            ylabel=Validation Accuracy,
            xticklabels={,,},
            ymin=0.3,
            legend pos = south east,
            xmin=-10]
            \addplot[mark=None, color=red] 
                table[x=Step, y=Value, col sep=comma]{images/network_csv/scheduler/MobileNetV2/lr1/MobileNetV2_scheduler_cosine_lr1_validation_acuracy.csv};
            \addplot[mark=None, color=green] 
               table[x=Step, y=Value, col sep=comma]{images/network_csv/scheduler/MobileNetV2/lr1/MobileNetV2_scheduler_cosine_distance_lr1_validation_acuracy.csv};
                

            \nextgroupplot[
                grid=major, 
                grid style={dashed,gray!30},
                % xlabel=Epoch,
                ylabel=Distance,
                yticklabel pos=right,
                xticklabels={,,},
                ylabel near ticks]
            \addplot[mark=None, color=red] 
                table[x=Step, y=Value, col sep=comma]{images/network_csv/scheduler/MobileNetV2/lr1/MobileNetV2_scheduler_cosine_distance_lr1_distance0.csv};
            \addplot[mark=None, color=green] 
                table[x=Step, y=Value, col sep=comma]{images/network_csv/scheduler/MobileNetV2/lr1/MobileNetV2_scheduler_cosine_lr1_distance0.csv};

        

            \nextgroupplot[
            grid=major, 
            grid style={dashed,gray!30},
            xlabel=Epoch, % Set the labels
            ylabel=Validation Accuracy,
            ymin=0.5,
            xmin=-10]
            \addplot[mark=None, color=red] 
                table[x=Step, y=Value, col sep=comma]{images/network_csv/scheduler/ResNet32/lr1/ResNet32_scheduler_cosine_lr1_validation_acuracy.csv};
            \addplot[mark=None, color=green] 
                table[x=Step, y=Value, col sep=comma]{images/network_csv/scheduler/ResNet32/lr1/ResNet32_scheduler_cosine_lr1_distance_validation_acuracy.csv};
        

            
            \nextgroupplot[
                grid=major, 
                grid style={dashed,gray!30},
                xlabel=Epoch, % Set the labels
                ylabel=Distance,
                yticklabel pos=right,
                ylabel near ticks]
            \addplot[mark=None, color=red] 
                table[x=Step, y=Value, col sep=comma]{images/network_csv/scheduler/ResNet32/lr1/ResNet32_scheduler_cosine_lr1_distance0.csv};
            \addplot[mark=None, color=green] 
                table[x=Step, y=Value, col sep=comma]{images/network_csv/scheduler/ResNet32/lr1/ResNet32_scheduler_cosine_lr1_distance_distance0.csv};

            \end{groupplot}
        \end{tikzpicture}
        \caption{Validation accuray and Distance for MobileNetV2 (upper) and ResNet32 (lower) trained with learning rate schedulers and a learning rate of 0.1.}
    \end{center}
\end{figure}



\section{ensemble methods}\label{res:Ensemble}
Ensemble methods were introduced in section \ref{sub:Ensemble_Methods}, the idea
beeing that the combined prediction of multiple networks would result in better
performance. However, the networks need uncorellated errors for that
increasement.

In the approach from \cite{loshchilov2016sgdr}, they took a snapshot of the
network at the end of each cycle. This procedure results in a performance boost
of around 1\% against a standard consine decay from section \ref{res:Scheduler}
without any additional training cost for MobileNetV2. The high learning rate after the warm
restart ensures that the each new snapshot distances itself in the parameter
space from the old. The hope is, that this distance results in uncorellated
errors.

As motivated in section \ref{sub:Motivation}, we try to increase that distance
more explicitly with our distance function. We use the same network as in
section \ref{res:Multiple} and take new snapshots before adding a new
checkpoint. After taking the snapshot, we add it to the ensemble, which consists
of all previous snaphsots and the model that is currently trained on. For the
ensemble prediction, we use the model averaging technique from
\ref{sub:Ensemble_Methods}.


For MobileNetV2, this leads to a ensemble accuracy of 94\%, which is 4\% better
than the validation accuracy of the single network. Figure \ref{fig:Ensemble}
shows a stepwise increase in ensemble accuracy after 150, 300, 450 and 600
epochs, when a snapshot is added to the ensemble. This shows that each new
network adds performance to the ensemble. However, the network doesn´t reach the
performance of cosine decay with warm restart and is even similar to a standard
network without any decay or distance function. The difference is probably due
to the fact, that cosine decay reaches a better performance for each snapshots
of the ensemble. The predictions may therefore be better in overall, even if
they are not more versatile.

To see if a ensemble would benefit from snapshots that have a further distance
between each other and therefore have a lower corellation, we increase
the strength parameter. However, the ensemble accuracy doesn't benefit from this.
Instead, the accuracy gets even worse for $s=10$ and $s=100$. 

In order to check if further distance truly means more different predictions, we
compare the individual predictions of the networks. Table \ref{tab:Ensemble}
shows the results. If we compare the different strength values of the distance
function, we can see that the number of examples for which the predictions of
the networks is the same has no difference between the strength values. This
suggests that although the snapshots of the ensembles differ in distance to each
other, this doens't result in different predictions. Even compared to the
baseline, the coherence seems similar. One possible explanation could be the
weight space symetry of section \ref{sub:Local_minima}, where due to symetry in
weights, networks can be distnant in weight space but still have the same
predictions for all training examples. Another possible explanation could be the
chracteristics of the training set, which may contain some easy and some
difficult examples. As a fixed learning rate doesn't exploit the landscape
enough, all networks may get the same easy examples right, regardless of their
position on the loss landscape. But for hard training examples, all these
networks fail.

\begin{table}[H]
    \centering
    \begin{tabular}{c|c|c|c}\label{tab:Ensemble}
        baseline & cosine & $s=10$ & $s=100$ \\
        \hline
        7648 & 9130 & 7739 & 7788 \\
    \end{tabular}
    \caption{Number of test examples where all snapshots agree in their classification for different configurations of MobileNetV2.}
\end{table}



For cosine decay, the coherence is even higher. This seems reasonable, as
snaphsots which perform better necessarly need to have more similar predictions,
as there are less options where they are wrong and can therefore disagree.
Nevertheless it is remarkable that although the distance between the networks
for cosine decay and a fixed learning rate is quite similar, the prediction
coherence differs strongly. This further suggests that a distance in parameter
space doens't result in different behaviour. Cosine decay probably exploits the
local landscape better due to a low learning rate and therefore arrives at more
similar snapshots.

\begin{figure}[h]\label{fig:Ensemble}
    \begin{center}
        \begin{tikzpicture}
            \begin{groupplot}[
                group style={
                group size=2 by 2,
                horizontal sep=10pt,
                group name=G},
                width=8cm,
                ymin=0.85,
                ymax=0.97
            ]
                
            \nextgroupplot[
            grid=major, 
            grid style={dashed,gray!30},
            ylabel=Ensemble accuracy,
            ylabel near ticks,
            legend pos=south east
            ]
            \addplot[mark=None, color=red] 
                table[x=Step, y=Value, col sep=comma]{images/network_csv/ensemble/MobileNetV2/MobileNetV2_ensemble_baseline_ensemble_accuracy.csv};
                \addlegendentry{baseline}
            \addplot[mark=None, color=green] 
                table[x=Step, y=Value, col sep=comma]{images/network_csv/ensemble/MobileNetV2/MobileNetV2_ensemble_baseline_cosine_ensemble_accuracy.csv};
                \addlegendentry{cosine}
            \addplot[mark=None, color=blue] 
                table[x=Step, y=Value, col sep=comma]{images/network_csv/ensemble/MobileNetV2/MobileNetV2_ensemble_distance_f10_ensemble_accuracy.csv};
                \addlegendentry{$s=10$}
            \addplot[mark=None, color=orange] 
                table[x=Step, y=Value, col sep=comma]{images/network_csv/ensemble/MobileNetV2/MobileNetV2_ensemble_distance_f100_ensemble_accuracy.csv};
                \addlegendentry{$s=100$}
            
            \nextgroupplot[
                    grid=major, 
                    grid style={dashed,gray!30},
                    ylabel=Validation Accuracy,
                    yticklabel pos=right,
                    ylabel near ticks]
            \addplot[mark=None, color=red] 
                table[x=Step, y=Value, col sep=comma]{images/network_csv/ensemble/MobileNetV2/MobileNetV2_ensemble_baseline_validation_acuracy.csv};
            \addplot[mark=None, color=green] 
                table[x=Step, y=Value, col sep=comma]{images/network_csv/ensemble/MobileNetV2/MobileNetV2_ensemble_baseline_cosine_validation_acuracy.csv};
            \addplot[mark=None, color=blue] 
                table[x=Step, y=Value, col sep=comma]{images/network_csv/ensemble/MobileNetV2/MobileNetV2_ensemble_distance_f10_validation_acuracy.csv};
            \addplot[mark=None, color=orange] 
                table[x=Step, y=Value, col sep=comma]{images/network_csv/ensemble/MobileNetV2/MobileNetV2_ensemble_distance_f100_validation_acuracy.csv};


            \nextgroupplot[
                    grid=major, 
                    grid style={dashed,gray!30},
                    xlabel=Epoch, % Set the labels
                    ylabel=Ensemble accuracy,
                    ylabel near ticks,
                    legend pos=south east,
                    ymax=0.95
                    ]
                    \addplot[mark=None, color=red] 
                        table[x=Step, y=Value, col sep=comma]{images/network_csv/ensemble/ResNet32/ResNet32_ensemble_baseline_ensemble_acuracy.csv};
                    \addplot[mark=None, color=green] 
                        table[x=Step, y=Value, col sep=comma]{images/network_csv/ensemble/ResNet32/ResNet32_ensemble_baseline_cosine_ensemble_acuracy.csv};
                    \addplot[mark=None, color=blue] 
                        table[x=Step, y=Value, col sep=comma]{images/network_csv/ensemble/ResNet32/ResNet32_ensemble_distance_ensemble_acuracy.csv};
                    %\addplot[mark=None, color=orange] 
                     %   table[x=Step, y=Value, col sep=comma]{images/network_csv/ensemble/ResNet32/ResNet32_ensemble_distance_f100_ensemble_acuracy.csv};
                    
            \nextgroupplot[
                            grid=major, 
                            grid style={dashed,gray!30},
                            xlabel=Epoch, % Set the labels
                            ylabel=Validation Accuracy,
                            yticklabel pos=right,
                            ymax=0.95,
                            ylabel near ticks]
                    \addplot[mark=None, color=red] 
                        table[x=Step, y=Value, col sep=comma]{images/network_csv/ensemble/ResNet32/ResNet32_ensemble_baseline_validation_acuracy.csv};
                    \addplot[mark=None, color=green] 
                        table[x=Step, y=Value, col sep=comma]{images/network_csv/ensemble/ResNet32/ResNet32_ensemble_baseline_cosine_validation_acuracy.csv};
                    \addplot[mark=None, color=blue] 
                        table[x=Step, y=Value, col sep=comma]{images/network_csv/ensemble/ResNet32/ResNet32_ensemble_distance_validation_acuracy.csv};
                    %\addplot[mark=None, color=orange] 
                     %   table[x=Step, y=Value, col sep=comma]{images/network_csv/ensemble/ResNet32/MobileNetV2_ensemble_distance_f100_validation_acuracy.csv};


            \end{groupplot}
            
        \end{tikzpicture}
        \caption{Ensemble accuracy of the snapshot ensemble (left) compared to the validation accuracy of the network where snapshots are taken from (right). [fehlt noch ein s=100 für ResNet32] }
    \end{center}
\end{figure}


\section{Computational Cost}\label{Res:Computational_cost}

Improvement in performance often arises with an increase in computational cost.
In Section \ref{sub:Computational_Analysis} this was discussed in a theoretical
setting, suggesting that the additional cost would scale linear with each
checkpoint. If we analyze the baseline network without distance from section
\ref{res:Baseline}, we reach a per epoch training time of 72s for MobileNetV2
and 55s for ResNet32, trained on the TCML Cluster. For the network trained with
distance function, we start with the same epoch time, as the analysis expected.
That`s due to the fact, that we also start training without the distance
function. But once we add a checkpoint to MobileNetV2, the epoch time rises by
around 22s to a total of 92s, see figure \ref{fig:Epoch_time}. If we add
multiple checkpoints, the increase for each stays constant at around 21ms. For
ResNet32, the amount that adds for each checkpoint is more variable. Here, the
first adds 8s, the second 12s and the third 10s. Nevertheless, the hypothesis of
the theretical computational cost seems to get confirmed. MobileNetV2 shows a
linear increase for each checkpoint. Even if the results are more variable for
ResNet32, it seems unlikely that the increase is superlinear, especially because
the last checkpoints adds less time again. The instability is probably due to
changes in hardware performance such as high temperature leading to a reduction
in clock speed.

\begin{figure}[h]\label{fig:Epoch_time}
    \begin{center}
        \begin{tikzpicture}
            \begin{groupplot}[
                group style={
                group size=2 by 1,
                horizontal sep=10pt,
                group name=G},
                width=8cm,
            ]
                
            \nextgroupplot[
            grid=major, 
            grid style={dashed,gray!30},
            xlabel=Epoch, % Set the labels
            ylabel=Epoch time,
            ylabel near ticks,
            legend pos=north west,
            legend style={nodes={scale=0.7, transform shape}}
            ]
            \addplot[mark=None, color=red] 
                table[x=Step, y=Value, col sep=comma]{images/network_csv/baseline/MobileNetV2/MobileNetV2_baseline_epoch_time.csv};
                \addlegendentry{baseline}
            \addplot[mark=None, color=green] 
                table[x=Step, y=Value, col sep=comma]{images/network_csv/baseline/MobileNetV2/MobileNetV2_baseline_distance_epoch_time.csv};
                \addlegendentry{one checkpoint}
            \addplot[mark=None, color=blue] 
                table[x=Step, y=Value, col sep=comma]{images/network_csv/multiple/MobileNetV2/run-mobileNetV2_multiple_3-tag-train_epoch_time.csv};
                \addlegendentry{multiple checkpoints}
            
            \nextgroupplot[
            grid=major, 
            grid style={dashed,gray!30},
            xlabel=Epoch, % Set the labels
            ytick pos=right
            ]
            \addplot[mark=None, color=red] 
                table[x=Step, y=Value, col sep=comma]{images/network_csv/baseline/ResNet32/ResNet32_baseline_epoch_time.csv};
            \addplot[mark=None, color=green] 
                table[x=Step, y=Value, col sep=comma]{images/network_csv/baseline/ResNet32/ResNet32_baseline_distance_epoch_time.csv};
            \addplot[mark=None, color=blue] 
                table[x=Step, y=Value, col sep=comma]{images/network_csv/multiple/ResNet32/ResNet32_multiple_epoch_time.csv};
            

            \end{groupplot}
            
        \end{tikzpicture}
        \caption{Epoch time in seconds for different configurations of MobileNetV2 (left) and Resnet32 (right).}
    \end{center}
\end{figure}

