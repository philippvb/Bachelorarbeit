\chapter{Summary and Outlook}

In the beginning we have seen that the setting for optimization is very
challenging. A highly non-convex loss landscpape which is expensive to compute
and consists of a large number of local extrema. In practice however,
optimization becomes suprisingly easy, even simple optimizers like SGd reaching
good results. Current literature suggest that this may be due to that local
minima mostly have low loss, exist everywhere on the landscape and are some
connected by paths of low loss. However, it appears that these connected minima are
concentrated in enclosed areas surrounded by arewallsas of high loss.

The motivation was to distance from a checkpoint on the landscape to overcome
these boundaries and explore the landscape further. In earlier work, Cosine
Decay with Warm restart achieved this by increasing the learning rate again,
which in turn lead to an increase in performance. In this work, the exploration
was forced more directly by adding a distance term to the loss function, which
penalizes small distances to the checkpoint. The results showed that this lead
to a larger distance compared to normal SGG or cosine decay. Furthermore the
distance could be controlled by the modification of the Kernel of the loss
function. However, a further distance didn't increase the validation accuracy.
This lead to the conclusion, that most local minima have approximately the same
accuracy regardless their position on the landscape.

Adding multiple checkpoints however lead to an increase in performance. The
pattern that arised look similar to cosine decay with warm restart. It was
argued that this might be due to similar underlying effects on the parameter
updates. Further work in this area should focus on better understanding the
reasons for the increase, especially why warm restarts tend to find better
minima which normal SGD fails to find.

A second effect was found for a suboptimal initial learning rate in combination
with Cosine Decay with Warm Restart. Here, the addition of the distance function
stabilized the training process. As this was limited to a particular learning
rate, there was no trivial explanation to that, but also did not seem to arise
by chance, as both MobileNetV2 and ResNet32 showed the same behaviour.
Therefore, further work should focus on explaining this effect and finding other
examples where the distance function could be beneficial.

Finally, the distance function was applied to ensemble methods, with the idea of
getting more uncorellated snapshots from a larger distance on the loss
landscape. However, the corellation seemed to stay the same regardless of the
distance. These results question the underlying argument for snapshot ensembling
in the case of Cosine Decay with Warm Restart, where a high learning rate was
used to try to create uncorellated snapshots. In general, the results questions
the connection between the loss landscape and the behaviour of network, as
different points on the loss landscape did not mean different predictions.
Therefore, the connection between loss landscape and behaviour has to be further
investigated. 
