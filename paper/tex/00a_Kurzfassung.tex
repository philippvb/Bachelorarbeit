Dieses Dokument soll zeigen, wie mit \LaTeXe{} eine Abschlussarbeit entsprechend
der Richtlinien am Lehrstuhl Rechnerarchitektur erstellt werden kann. Es handelt
sich hierbei weder um eine Einführung in \LaTeX, noch in wissenschaftliche
Methode oder Schreibweise. Es werden einige Stilelemente und häufige
Schwierigkeiten exemplarisch herausgegriffen und vorgestellt. Ferner soll kurz
Aufbau und Benutzung dieser Vorlage beschrieben werden. Weiterführende Literatur
ist im Anhang aufgeführt. Zu den hier dargestellten \LaTeX-Paketen existieren
weiterhin eigene Dokumentationen, die nicht alle im Literaturverzeichnis
aufgeführt sind. Diese Vorlage basiert auf dem \KOMAScript-Paket
\cite{Neukam2003}, da hier die Dokumentenklasse Report (\verb!scrreprt!)
verwendet wird. Wichtig ist, dass die getätigten Einstellungen in diesem
Dokument nicht verändert werden, außer wenn statt in deutscher lieber in
englischer Sprache geschrieben wird. Dazu sind bereits einige Pakete vorgesehen,
die aber z.\,T. mit Kommentaren versehen und somit nicht aktiv sind. Die Datei
\verb!Vorlage.tex! dient als Hauptdokument. Der eigentliche Text sollte aber in
einer eigenen Datei pro Kapitel geschrieben werden, die dann lediglich in das
Hauptdokument eingebunden werden müssen. Dieses Vorgehen wird mit einem Kapitel
beispielhaft dargestellt. Ebenso können vorhandene Quelltexte direkt in das
Dokument integriert oder als Pseudocode aufgelistet werden. Auch wird
beispielhaft die Verwendung von Unterabbildungen, langen und normalen Tabellen
sowie mathematischen Formeln gezeigt. Abschließend wird kurz auf weiterführende
Literatur eingegangen. Ebenso beispielhaft werden im Abkürzungsverzeichnis
einige Einträge vorgenommen. Generell sollen nur verwendete Abkürzungen, die
nicht im Duden enthalten sind, in dieses Verzeichnis aufgenommen werden.