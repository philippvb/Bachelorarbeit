\chapter{Dicussion}

In section \ref{sub:Motivation}, we summarized some of the challenges of
optimization and described our approach to overcome some of these. The key idea
was to push the network away from a checkpoint to escape local minima and
explore the loss landscpae further. The following sections described how we
tried to realize this goal: We expanded the loss function by a term which
measures the distance between the checkpoint and current state and penalizes
small distances.

The results showed that the distance function acted successfull at achieving
this goal. Whenever we included the distance term, the network distanced further
than without, due to the additional gradient. At the beginning, there was a
larger increase in distance due to the large gradient of the RBF Kernel of the
distance function. When the gradient becomes small enough in later epochs, the
distance converges. In section \ref{res:Hyperparameters} we showed, how this
distance can be modified. A wider distance function lets the network distance
further, but at a slower rate. This reflects the shape of the RBF Kernel,
because a larger width factor lets the function become wider, but also have a
smaller gradient, see figure \ref{fig:Gaussian}. A larger gradient can be
created by varying the strength factor. This results in the same effect as
above, but at a much faste rate, because of the larger gradient. In summary, the
distance behaves like we would predict from the shape of the distance function.


Another interesting fact was that even without a distance Kernel,
SGD also increases the distance, even when the validation accuracy seems to
converge. This gives evidence for another difference between neural networks and
traditional optimization. Where in traditional optimization, the goal normally
is to reach global optimum, optimization in neural networks fails to do so.
Nevertheless, the good performance of optimizer like SGD shows, that it may not
even be necesarry or useful to search for a global optimum. The continous motion
of SGD also implies that traditional termination metrics like 0 gradients are
not sufficient, as the size of the gradients doesn't decay over time. Therefore,
other metrics like early stopping at a good validation accuracy have to be taken
into account, as discussed in section \ref{sub:Stopping_criteria}.
\newline

Recent research on loss landscape suggests that there are cells of local minima
which are connected by valleys of low loss. Between these cells however lie
areas of high loss. To overcome these areas of high loss and jump to other
cells, we used our distance function. Whenever we created a checkpoint, we saw a
drop in validation accuracy. The higher the influence of the Kernel, the higher
the drop was. Section \ref{res:Hyperparameters} showed some reasons for this
from a mathematical viewpoint. Whenever the influence of the distance function
is higher, its gradient will also be larger compared to the gradient of the
cross-entropy-loss. Therefore, the gradient of the cross-entropy-loss will be
neglected, we don't follow the path of low loss any more and the accuracy drops,
as a minima is surrounded by areas of high loss.

However, after further epochs, the accuracy recovered to the same level as
before. Combined with the distancing, this means that areas of low loss can be
found everywhere on the landscape. This is consistent with current literature
like in \cite{he2020piecewise}. But in most cases, the network with distance
function doesn't top the network without. This means that even if there are
areas of low loss everywhere on the landscape, no area seems to be much better
than the other. Therefore, distancing is possible, but doesn't seem to be
useful. If this is the case, the question if the large exploration of the loss
landscape is useful also arises, because a new area seems to not be able to
achieve better performance. 
\newline

For multiple checkpoints however, there is a small increase with distance
function compared to without. Furthermore, the same pattern as in cosine decay
arises, where new cycles top the maximum accuracy of the previous. In section
\ref{sub:Effect_on_Gradient} we suggested that this might relate to the simliar
influence on the gradient, that the parameter updates just get larger at the
beginning. An open question is still why this leads to a boost in accuracy. Here
we will argue, that a high learning rate might lead to wider minima and
therefore a better generalization.

Consider the case where the networks after some training has a high performance
and seems to be in an area of converge. This means that most of the parameters
are in a local minima, surrounded by areas of higher loss, though not all of
them, as we have seen existing valleys. If we perform a warm restart, the
inaccuracy in the gradient from batch methods for example will be amplified by
the large learning rate. Consequently, the local minimum of the parameters will
be overshoot from both sides and the parameter values will distance further from
their local minimum. As the minima are surrounded by a high loss, the accuracy
drops as we have seen in the results, section \ref{res:Scheduler}. But if a
parameter still returns to the same minimum as before, the minimum is robust in
the sense that the surrounding boundaries of large loss are high and wide. For
parameters whose local minima is very narrow and another area of low loss is
nearby, the values may switch to these dueto high learning rate. Therefore,
parameters which have reached good minima will stay in their position, while
unstable minima may be left. Here also lies a key difference to the distance
function. As the distance term acts the same on all parameters, the distance
will be increased to both good and bad parameter values. This could provide an
explanation for a smaller boost in accuracy of 0.8\% compared to 1\% for cosine
decay.
\newline

One major improvement of the distance function was for a wrong initial learning
rate. At first, it seems that a cosine decay should let the network be more
robust to a wrong learning rate because it is decayed until 0, irrelevant of the
inital learning rate. In section \ref{res:Learning_rate}, this was the case
until the first warm restart. But after it was performed, both ResNet32 and
MobileNetV2 started diverging for a initial learning rate of 0.01 in a way that
the maximum learning rate of each new cycle decreased compared to the previous
one. In contrast when adding a distance function, the maximum accuracy stayed
constant for MobileNetV2, and decreases significantly less for ResNet32.

A simple explanation would be, that cosine decay with warm restart walks into an
area of high loss, and is not able to recover. The distance function then helps
to distance from this point and return to areas of lower loss, leading to a
stable performance. The distance plot of figure \ref{fig:Results_wrong_lr}
conformed to this idea, as normal cosine decay was not able to distance from the
checkpoint, in contrast to configurations with the optimal learning rate.
However, this would contradict both our results and most papers. As the distance
function had no impact on the valdiation accuracy in section \ref{res:Baseline},
we suggested that this happens due to the fact, that areas of low loss exist
everywhere, which would contradict this explanation. Furthermore, it is unlikely
that both networks would walk into an area of high loss by chance. Because the
effect also only happens for this particular learning rate, the truecuase of the
difference remains unclear.
\newline

In section \ref{res:Ensemble}, we saw how the performance could be improved by
network ensembling. As section \ref{sub:Ensemble_Methods} discussed, there are
two major factors which influence the performance of the ensemble. The
individual performance of the network and the corellation between the network
predictions.

The results showed that although the corellation is an important factor, a low
corellation can not recover if the network performs worse. It seems like a low
corellation leads to much larger benefits of the ensemble in comparison to the
single network. Cosine decay had a much larger corellation and this resulted
only in a small increase of 1\%, where for our distance function, the
ensemble gained over 4\% in accuracy against the single network. However, cosine
decay still performs much better when compared to our network due to the better
single network performance. One further aspect to keep in mind is that a better
network performance automatically means higher corellation. Consider a network
which performs at 100\% accuracy. Then all its snapshots have a corellation of
1, since all of them predict all exampels right. If networks have high but not
perfect accuracy, there still has to be a high corellation, since there are not
mayn examples where the snapshots predict wrong and can disagree. Therefore, a
higher corellation often automatically arises and may rather predict the
relative gain to a single network than the absolute performance.

For networks with lower accuracy, we tried to reduce the corellation by using
our distance function. Theoretically, the ensemble should then perform better,
since we have seen in the other results that a larger distance doesn't harm the
accuracy, at least in long term. However, the results showed that even for
larger distances, the corellation between the snapshots stayed the same. The
implications for the loss landscape remain unclear.

At first, the results seem to suggest that networks with same accuracy predict
the same regardless their position on the loss landscape. One explanation for
this could be the weight space symetry, as discussed in section
\ref{sub:Local_minima}. Here, networks perform the same due to the symetry in
weight space, which arises from swapping two units. If the snapshots are taken
at these points, they perform the same even if their difference is large.
However, it seems unlikely that the network converges exactly to this points
after distancing from their last snapshot. If this is the case, the question
arises if there may only be one truly optimal configuration for network, which
can be found at different locations in weight space.

Alternatively, the robust corellation could also be explained by the dataset. Maybe
there are examples, which are very hard. Therefore, the network has to exploit
the loss landscape very good in order to get correct predictions. As a fixed
learning rate fails to do so, these examples remain wrong regardless of the
networks position in weight space. As a result, the corellations stays high due
to the samne wrong predictions of every snapshot.

No matter the explanation, the results question the underlying idea of snapshot
ensembles. A high learning rate increases the distance, but doesn't result in a
lower corellation. Therefore, the benefits may be doubtful from a theoretical
viewpoint. Nevertheless, in practice our network with suboptimal accuracy
achieved a much higher accuracy as ensemble. 
\newline

One aspect to keep in mind is the additional training cost. Section
\ref{Res:Computational_cost} showed that the distance function adds a
significant amount to the epoch time. Even though this scales linear with every
new checkpoint the cost is very large, especially considering small amount of
improvement for most cases. Consequently if training performance is of
importance, it seems that other methods like a cosine decay reach better
improvement with no additional training cost. Nevertheless, the distance
function stabilized the choice of the learning rate for cosine decay. If
training time is not important, the distance may be worth to try out, especially
because it adds no cost at test time.


